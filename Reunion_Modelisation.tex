\documentclass{beamer}

% Vary the color applet  (try out your own if you like)
%\colorlet{structure}{red!65!black}
\usetheme{Singapore}
%\beamertemplateshadingbackground{yellow!100}{white}

\setbeamersize{text margin left=4mm, text margin right=4mm} 

%\usepackage{beamerthemesplit}
\usepackage{graphics}
\usepackage{graphicx}
\usepackage{hyperref}
\usepackage{multimedia}
\usepackage{hyperref}
\usepackage{multimedia}

\usepackage[T1]{fontenc}
\usepackage[utf8]{inputenc} % pour les accents (mettre latin1 pour
% windows au lieu de utf8)
\usepackage[frenchb]{babel} % le documents est en francais
\usepackage{amsmath} % un packages mathematiques
\usepackage{amssymb}				% collection de symboles math�matiques
\usepackage{amsthm}
\usepackage{xcolor} % pour definir plus de couleurs
\usepackage{graphicx} % pour inserer des figures

\setbeamertemplate{blocks}[rounded][shadow=true]
\setbeamercolor{block body}{fg=white,bg=red} 
\setbeamercolor{block body alerted}{fg=white,bg=red}

\def\hilite<#1>{%
  \temporal<#1>{\color{gray}}{\color{blue}}%
               {\color{blue!25}}}


\title[Réunion Modélisation]{Mélange vertical profond: une révolution}
\author[Pierre Labreuche]{Pierre Labreuche }%\inst{1}}
\institute[LEGI:]{
%  \inst{1}%
   \'Equipes MEOM \& MEIGE
   \\LEGI}
\date[\today]{}%Présentation de stage}
\subject{Géophysique}

%\pgfdeclaremask{fsu}{fsu_logo_ybkgrd}
%\pgfdeclareimage[mask=fsu,width=1cm]{fsu-logo}{fsu_logo_ybkgrd}

%\logo{\vbox{\vskip0.1cm\hbox{\pgfuseimage{fsu-logo}}}}

\AtBeginSubsection[]
{
  \begin{frame}<beamer>
    \frametitle{Où en est on?}
    \tableofcontents[currentsection,currentsubsection]
    \begin{figure}[htbp]
%    \includegraphics[height=2cm]{Plots/mechanisme.png}
    \end{figure}

  \end{frame}
}



\begin{document}

\frame
{
\titlepage
%\includegraphics[width=2cm]{images/logoens} \hfill \includegraphics[width=2cm]{images/logoucbl} \hfill \includegraphics[width=2cm]{images/logounivlyon}  \hfill \includegraphics[width=2cm]{images/logolegi}

}
%Table of Topics

\section{Cours 101}

\subsection{Les ondes}

\frame
{
Les ondes internes (IW):
\begin{itemize}
\item<1-> Générées par le vent, ou l'action conjointe des marées ou d'un courant profond avec de la bathymétrie
\item<1-> Très énergétiques dans l'océan
\item<1-> Un des principaux signaux sous-mésoéchelle (mais pas que)
\item<1-> De fréquence comprise entre f et N
\begin{figure}[htbp]
%\uncover<2->{ \includegraphics[height=5cm]{Plots/Snapshot_highlight.jpg} \\}
\end{figure}
\end{itemize}
}

\frame
{
Les oscillations inertielles (IO):
\begin{itemize}
\item<1-> Un cas dégénéré d'onde interne de fréquence f
\item<1-> D'échelle horizontale bien plus grande que les autres ondes internes ($\sim 10 - 100 km$) (homogène selon l'horizontale pour mon étude)
\end{itemize}
}

\subsection{Mechanisme}
\frame
{
\begin{figure}[htbp]
%\uncover<2->{
%\alt<2>{ \includegraphics[height=3cm]{Plots/topo_U_G.png} } {
%\alt<3>{ \includegraphics[height=3cm]{Plots/topo_U_G_ILW.png} } {
%\alt<4>{ \includegraphics[height=3cm]{Plots/topo_U_G_ILW_IO.png} } {
%{ \includegraphics[height=3cm]{Plots/mechanisme.png}} }}}
%}
\end{figure}
\uncover<1->{Le schéma qui explique le méchanisme de Nikurashin et Ferrari: \\}
\uncover<2->{Un courant géostrophique passe au dessus d'une topographie\\}
\uncover<3->{Le courant crée des ondes \\}
\uncover<4->{Les ondes déferlent et créent des oscillations inertielles\\}
\uncover<5->{Les oscillation inertielles augmentent le déferlement des ondes internes\\}
\uncover<6->{... Etc etc ...}
}

%\frame
%{
%\tableofcontents
%}

\section{Des gros calculs}
\subsection{Les équations}
\frame
{
On adimensionne les équations par rapport à la raideur des ondes internes:
\alert{$$\epsilon = k_z h_T \sim \frac{N h_T}{U_G}$$}
Avec une dissipation de Raigleigh: $D(\bf{u}) = -\lambda \bf{u}$\\
\uncover<2->{
Et on fait un dévelloppement multi-échelle (Bender-Orzag):
\tiny{
%\begin{align*}
$$
t = t + \epsilon^{-1}T^{(1)} + \epsilon^{-2} T^{(2)} + \cdots$$
$${\bf x} = {\bf x} + \sum_{i=1}^{\infty} \epsilon^{-i}{\bf X}^{(i)}
$$%%\end{align*}
}
pour les coordonnées, et pour les variables: \\
\tiny{
%\begin{align*}
$$\bf{u} = \bf{u}^G(T_G,X_G) + \bf{u}^I(t,T_I,X_I) + \sum_{i=1}^{\infty} \epsilon^i \bf{u}^{(i)}$$
$$w = w^G(T_G,X_G) + w^I(t,T_I,X_I) + \sum_{i=1}^{\infty} \epsilon^i w^{(i)}$$
%p &= p^G(T_G,X_G) + \epsilon^2 Ro p^I(t,T_I,X_I) + Ro \sum_{i=1}^{\infty} \epsilon^i u^{(i)}\\
%b &= b^G(T_G,X_G) + \epsilon^3 b^I(t,T_I,X_I) + \sum_{i=1}^{\infty} \epsilon^i b^{(i)}\\
%D_{m,b} &= \epsilon D_{m,b}^{(1)} + \epsilon^2 D_{m,b}^{(2)} + \cdots
%\end{align*}
}}}
\frame
{
On développe, on fait plein de trucs cools et on a:
%\begin{align*}
%$$Nu = u + i v $$
\alt<1>{
$$\partial_t \overline{V^{(3)}} + i f \overline{V^{(3)}}  = - A + U_I k_T f^{-1}  C\quad e^{i f(t-t_0)}  - \lambda \overline{V^{(3)}}$$
%\end{align*}
%\begin{align*}
$$ - ( \partial_{T^{(3)}}U_{IO} + i f \partial_{T^{(3)}}t_0 U_{IO} - U_{IO} k_T f^{-1}  B) \quad e^{-i f(t-t_0)}$$
%\end{align*}


}
{
$$\alert{\partial_t \overline{V^{(3)}} + i f \overline{V^{(3)}} } = - A + U_I k_T f^{-1}  C\quad e^{i f(t-t_0)}  - \lambda \overline{V^{(3)}}$$
%\end{align*}
\alert{
%\begin{align*}
$$ - ( \partial_{T_{slow}}U_{IO} + i f \partial_{T_{slow}}t_0 U_{IO} - U_{IO} k_T f^{-1}  B) \quad e^{-i f(t-t_0)}$$
%\end{align*}
}
}
\uncover<2>{}
}
\subsection{Les résultats du calcul}
\frame
{
\color{magenta}{$$\partial_{T_{slow}}U_{IO} + i f \partial_{T_{slow}}t_0 U_{IO} - U_{IO} k_T f^{-1}  B = 0$$}
\color{black}
Evolution des oscillations inertielles:
$$U_{IO}(t,T_{slow}) = U_{IO}(0) \quad e^{\Gamma T_{slow}} \quad e^{f(t-t_0)}$$
\begin{overprint}
\onslide<1>
En séparant l'évolution de l'amplitude et de la phase:
$$
U_{IO}(T_{slow}) = U_{IO}(0) \quad e^{k_T f^{-1} \Re (B) T_{slow}}
$$
$$
t_0(T_{slow}) = k_T f^{-2} \Im (B) T_{slow}
$$

\onslide<2-3>


$$\Gamma \sim \alpha . \big( 2 cos(\frac{\pi z}{h_c}) + \frac{\lambda}{f}sin(\frac{\pi z}{h_c})  \big)e^{-\frac{\lambda}{f} \frac{\pi z}{h_c}}$$

$$t_0    \sim \alpha . \big( 2 sin(\frac{\pi z}{h_c}) - \frac{\lambda}{f}cos(\frac{\pi z}{h_c})  \big)e^{-\frac{\lambda}{f} \frac{\pi z}{h_c}} \quad T^{(3)}$$
\end{overprint}

\uncover<3->{
%La vitesse de groupe des ondes internes:
%$$
%c_g^z(IW) = \frac{\partial \omega}{\partial \bf{k}}|_z \sim \frac{U_G^2 k_T}{N}
%$$

%\alert{
\begin{center}
\begin{minipage}{50mm}
      \begin{alertblock}{}
            \begin{center}
                  \color{green}{$h_c = \frac{\pi c_g^z(IW)}{f} \sim \frac{\pi U_G^2 k_T}{N f}$}
                  $\quad$
                  \color{blue}{$\alpha =  1 + 4 \frac {f^2}{U_G^2 k_T^2} - 6 \frac{\lambda^2}{U_G^2 k_T^2}$}

%                  \textbf{Très important}
            \end{center}
      \end{alertblock}
\end{minipage}
\end{center}
%\begin{theorem}<+->
%\color{red}{$h_c = \frac{\pi c_g^z(IW)}{f} $}
%$\quad$ , $\quad$
%\color{blue}{$\alpha =  1 + 4 \frac {f^2}{U_G^2 k_T^2} - 6 \frac{\lambda^2}{U_G^2 k_T^2}$}
%\end{theorem}
}
}
\frame
{
\color{magenta}{$$\partial_{T^{(3)}}U_{IO} + i f \partial_{T^{(3)}}t_0 U_{IO} - U_{IO} k_T f^{-1}  B = 0$$}
\color{black}
Evolution des oscillations inertielles:
$$U_{IO}(t,T_{slow}) = U_{IO}(0) \quad e^{\Gamma T_{slow}} \quad e^{f(t-t_0)}$$
\begin{center}
\begin{minipage}{50mm}
      \begin{alertblock}{}
            \begin{center}
                  \color{green}{$h_c = \frac{\pi c_g^z(IW)}{f}  \sim \frac{\pi U_G^2 k_T}{N f}$}
                  $\quad $
                  \color{blue}{$\alpha =  1 + 4 \frac {f^2}{U_G^2 k_T^2} - 6 \frac{\lambda^2}{U_G^2 k_T^2}$}

%                  \textbf{Très important}
            \end{center}
      \end{alertblock}
\end{minipage}
\end{center}

\uncover<2->{\color<2->{black}{=> Profil des oscillations inertielles}} \uncover<3->{ => profil de dissipation}
%\end{overprint}
}

\section{Une petite comparaison}
\subsection{Le modèle}
\frame
{
Vérification des résultats:
\begin{columns}
\begin{column}[l]{6cm}
\begin{itemize}
\item<1-> Symphonie NH (le fameux)
\begin{itemize}
\hilite<2> \item En 2D, périodique en (x,y)
\hilite<2> \item Grille: 12.5m x 5m
\hilite<2> \item Domaine: 2km x 2km
\hilite<2> \item En haut: couche éponge (viscosité augmentée)
\hilite<2> \item Forçage sur $\frac{\partial P}{\partial y}$ pour $\bf{u} \rightarrow \bf{U}_G$
\end{itemize}
\item<3->Paramètres physiques
\begin{itemize}
\hilite<4> \item Bathy: $h_t . \sin(\frac{2.\pi x}{2 km})$
\hilite<4> \item $f=10^{-4}s^{-1}$
\hilite<4> \item $N=10^{-3}s^{-1}$
\hilite<4> \item $U_G = 0.1 m.s^{-1}$
\end{itemize}
\item<5-> Les paramètres qui changent:
\end{itemize}
\end{column}
\begin{column}[r]{6cm}
%\includegraphics[height=3cm]{Plots/Snapshot_highlight_cgz.png} \\
%\begin{itemize}
%\hilite<6> \item<6-> \includegraphics[height=2cm]{Plots/tableau_parametres.png}
%\end{itemize}
\end{column}
\end{columns}
}
\subsection{Comparaison Analytique/Modèle}
\frame
{
%\includegraphics[height=3cm]{Plots/Hovmoller_analytique_decay.jpg} 
%\includegraphics[height=3cm]{Plots/Comparaison_analytique_numerique_FS.jpg} 

\scriptsize{Hovmoller de $\Gamma(z) . \cos(f.t)$ à gauche, et Hovmoller des oscillations inertielles à droite}

Deux conclusions s'imposent à nous:
\begin{itemize}
\item<2->{Le profil supposé des oscillations inertielles, et notamment le noeud à \alert{$h_c$} semble correctement prédit}

\transdissolve<4>
\framezoom<3><4>[border](6.5cm,0cm)(3cm,3cm)
\item<5->{Les oscillations inertielles ont une vitesse de phase verticale égale à la vitesse de groupe des ondes internes: \color{blue}{$c_{\phi}^z(IO) = c_g^z(IW)$}}\color{black}
\end{itemize}
}

\subsection{Profils de dissipation}
\frame{
%\includegraphics[height=3cm]{Plots/Hovmoller_de_la_diffusion_turbulente_long_new_color%.jpg} \\
\transdissolve<3>
\framezoom<2><3>[border](0cm,0cm)(7cm,3cm)
%\includegraphics[height=3cm]{Plots/Energy_dissipation_rate_profile_color_new.jpg}
\transdissolve<6>
\framezoom<5><6>[border](0.2cm,3cm)(5cm,2.75cm)
\uncover<8->{Tout se passe bien en dessous de $h_c$ \\}
\uncover<9->{On s'attend et on observe une saturation de l'amplitude des oscillations inertielles et de la dissipation (not shown)}
}
\section{What next?}
  \begin{frame}<beamer>
    \frametitle{Où en est on?}
    \tableofcontents[currentsection,currentsubsection]
    \begin{figure}[htbp]
%    \includegraphics[height=2cm]{Plots/mechanisme.png}
    \end{figure}

  \end{frame}

\frame
{
\begin{itemize}
\item OSM 2012: fait
\uncover<2->{
\item Suite directe:
\begin{itemize}
\item Justifier le "phase-locking" par le calcul: \alert{à faire}
\item Expliquer la saturation: \alert{à faire}
\item Trouver un profil ET un taux de dissipation: \alert{à faire}
\end{itemize}
\item A grande échelle:
\begin{itemize}
\item Regarder la distribution de dissipation dans l'océan à partir de données: \alert{à faire}
\item Regarder l'impact de la dissipation sur la circulation à grande échelle: \alert{à faire}
\end{itemize}
\item Autre chemin énergétique:
\begin{itemize}
\item Regarder le dépot de qdm dans un écoulement géostrophique: \alert{à faire}
\end{itemize}
\item La terreur:
\begin{itemize}
\item Ecrire un article: \alert{à faire}
\end{itemize}
}
\end{itemize}
}
\frame
{
\begin{center}
\Huge{The End}
\end{center}
}
\frame
{
%\includegraphics[height=6cm]{Plots/Re_B.png}

}


\end{document}
