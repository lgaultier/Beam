%\documentclass[compress,notesonly]{beamer} %compress to make bars as small as possible
                                           %notesonly to add note not visible on screen (\note[]{})
\documentclass[compress,slidescentered,notes=show]{beamer}
%\documentclass[compress,slidescentered,notes=hide]{beamer}

%---------------FONT AND LANGUAGES ---------------------------
\usepackage[utf8]{inputenc}
%\usepackage[T1]{fontenc}
\usepackage[english]{babel}
\usepackage{pifont}

%--------------FORM ------------------------------
\usepackage{beamerthemesplit}
\usepackage{multicol} %possibility to create columns
\usepackage{comment} 
%\usepackage[absolute,showboxes, overlay]{textpos}
%\textblockorigin{1mm}{1mm}
%\TPshowboxestrue % or false to display contour

\usenavigationsymbolstemplate{}
\usetheme{Frankfurt} %Darmstadt}
\useoutertheme{smoothbars} %to add numerotation
\usepackage{color} %creation of own colors
\useoutertheme{infolines} %to add numerotation
%\usecolortheme[named=SeaGreen]{structure}
\definecolor{bleuclair}{rgb}{0.2,0.9,0.8}
%\definecolor{mycyan}{rgb}{.19,0.5,0.5}
\definecolor{mycyan}{rgb}{0.2,0.6,0.6}
\setbeamercolor*{palette primary}{use=structure,fg=white,bg=mycyan}
\setbeamercolor{block title}{bg=mycyan,fg=black}%bg=background, fg= foreground
\setbeamercolor{block body}{bg=lightgray,fg=black}%bg=background, fg= foreground
\setbeamercolor{structure}{bg=black, fg=mycyan}
\setbeamercolor{normal text}{fg=black}
\setbeamercolor{alerted text}{fg=red}
\setbeamercolor{frametitle}{fg=white}
\setbeamercolor{title}{fg=black}
\setbeamercolor{titlelike}{fg=black}
\setbeamercolor{title}{fg=black}
\setbeamercolor{section in sidebar}{fg=black}
\setbeamercolor{section in sidebar shaded}{fg= grey}
\setbeamercolor{subsection in sidebar}{fg=black}
\setbeamercolor{subsection in sidebar shaded}{fg= grey}
\setbeamercolor{itemize item}{fg=mycyan}
\setbeamercolor{section in tableofcontents}{fg=black,bg=black}
\setbeamercolor*{item projected}{fg = white, bg=mycyan} 
\useoutertheme{shadow}
%\setbeamercolor{sidebar}{bg=red}
%\beamertemplatetransparentcovered %set transparancy
\newcommand{\gu}[1]{#1}
%usepackage{default}
\usepackage{geometry} %put margin
\geometry{hmargin=0.25cm, vmargin=0.0cm}
\setbeamersize{text margin left=2mm, text margin right=2mm}
%\setbeamersize{text margin top=0cm}
%\setbeamersize{sidebar width left=0cm}
%\setbeamersize{sidebar width right=0cm}
%\usepackage{fullpage}
\setbeamertemplate{background}{\includegraphics[width=\paperwidth,height=\paperheight]{./pict/slide0006_background.png}}

%-------------GRAPHICS ----------------------------
\usepackage{graphicx} %add pictures
\graphicspath{{./pict/}{pict_nemo/}{/data2/gaultier/NEMO/DIAG/}} %path to pictures
\usepackage{tikz} %add schemes
\usetikzlibrary{shapes} %add diamonds shape to schemes
\usepackage{multimedia} %add video
\usepackage{array}
\pdfpageattr {/Group << /S /Transparency /I true /CS /DeviceRGB>>} %avoid color troubles using eps and acroread
\renewcommand{\figurename}{Fig.}
\setlength{\unitlength}{1cm} %to use pictures
\newcommand{\ftitle}[1]{\begin{center}\textcolor{mycyan}{#1} \end{center}} %title for figures
\newcommand{\legende}[1]{\textit{\footnotesize #1}}

%------------MAKE TITLE ----------------------------
\title{On the joint use of high resolution tracer images \hspace{4cm} and altimetric data for the control of ocean circulations.}
%\subtitle{\vspace{1.5em} \textbf{A Data Assimilation strategy}}
\author[LPO]{Lucile Gaultier, Jacques Verron, Pierre Brasseur, Jean-Michel Brankart}
\date{\textit{February 16, 2013}}

%\logo{\includegraphics[height=1.5cm]{./pict/logo_meom.jpeg}}
%\logo{\insertframenumber/\inserttotalframenumber}
%\pgfdeclareimage[height=1.2cm]{legi}{./pict/logo_legi.jpeg}
%\logo{\pgfuseimage{legi}}

\begin{document}

%1%%%%%%%%%%%%%%%%%%%%%%%%%%%%%%%%%%%%%%%%%%%%%%%%%%%%%%%%%%%%%%%%%%%%%
\begin{frame}
  \maketitle
\setbeamercolor*{palette primary}{use=structure,fg=white,bg=bleuclair}
  \begin{center}
    \includegraphics[height=1.5cm]{./pict/logo/logo_meom.jpeg}
    \hspace{0.5cm}
    \includegraphics[height=1.5cm]{./pict/logo/logo_legi.jpeg}
    \hspace{0.5cm}
    \includegraphics[height=1.5cm]{./pict/logo/logo_cnrs.jpeg}
    \hspace{0.5cm}
    \includegraphics[height=1.5cm]{./pict/logo/logo_cnes.jpeg}
  \end{center}

  \note{
}
\end{frame}
%%%%%%%%%%%%%%%%%%%%%%%%%%%%%%%%%%%%%%%%%%%%%%%%%%%%%%%%%%%%%%%%%%%%%%%%

\logo{\insertframenumber/\inserttotalframenumber}

\section{Context and objectives}
%-------------------------------%
        \subsection{Scales in the ocean}
%%%%%%%%%%%%%%%%%%%%%%%%%%%%%%%%%%%%%%%%%%%%%%%%%%%%%%%%%%%%%%%%%%%%%%%%
\begin{frame}

\end{frame}
%%%%%%%%%%%%%%%%%%%%%%%%%%%%%%%%%%%%%%%%%%%%%%%%%%%%%%%%%%%%%%%%%%%%%%%%

	\subsection[Observation]{Observation from space}
%%%%%%%%%%%%%%%%%%%%%%%%%%%%%%%%%%%%%%%%%%%%%%%%%%%%%%%%%%%%%%%%%%%%%%%%
\begin{frame}
%obs altimetric, 
%obs image
\end{frame}
%%%%%%%%%%%%%%%%%%%%%%%%%%%%%%%%%%%%%%%%%%%%%%%%%%%%%%%%%%%%%%%%%%%%%%%%

	\subsection{A data assimilation approach}
%%%%%%%%%%%%%%%%%%%%%%%%%%%%%%%%%%%%%%%%%%%%%%%%%%%%%%%%%%%%%%%%%%%%%%%%
\begin{frame}
%Def de l'AD
\end{frame}

\begin{frame}
%Objective
\end{frame}
%%%%%%%%%%%%%%%%%%%%%%%%%%%%%%%%%%%%%%%%%%%%%%%%%%%%%%%%%%%%%%%%%%%%%%%%

%%%%%%%%%%%%%%%%%%%%%%%%%%%%%%%%%%%%%%%%%%%%%%%%%%%%%%%%%%%%%%%%%%%%%%%%
\begin{frame}
  \frametitle{Outline}
  \tableofcontents%[pausesections]
\end{frame}
%%%%%%%%%%%%%%%%%%%%%%%%%%%%%%%%%%%%%%%%%%%%%%%%%%%%%%%%%%%%%%%%%%%%%%%%

\section[Method]{Combining altimetric data and tracer images}
%------------------------------------------------------------%

	\subsection[proxy FSLE]{Find a proxy between tracer and velocity}
%%%%%%%%%%%%%%%%%%%%%%%%%%%%%%%%%%%%%%%%%%%%%%%%%%%%%%%%%%%%%%%%%%%%%%%%
\begin{frame}
%A proxy is necessary
\end{frame}

\begin{frame}
%The FSLE is our proxy
%What is the FSLE
\end{frame}
%%%%%%%%%%%%%%%%%%%%%%%%%%%%%%%%%%%%%%%%%%%%%%%%%%%%%%%%%%%%%%%%%%%%%%%%

	\subsection[Inversion]{Inversion of the tracer image}
%%%%%%%%%%%%%%%%%%%%%%%%%%%%%%%%%%%%%%%%%%%%%%%%%%%%%%%%%%%%%%%%%%%%%%%%
\begin{frame}
%Inversion Method
\end{frame}

\begin{frame}
%DA Parameters )subspace of error, decreasing algorithm, Gibbs algorithm
\end{frame}
%%%%%%%%%%%%%%%%%%%%%%%%%%%%%%%%%%%%%%%%%%%%%%%%%%%%%%%%%%%%%%%%%%%%%%%%

%%%%%%%%%%%%%%%%%%%%%%%%%%%%%%%%%%%%%%%%%%%%%%%%%%%%%%%%%%%%%%%%%%%%%%%%
\begin{frame}
  \frametitle{Outline}
  \tableofcontents%[pausesections]
\end{frame}
%%%%%%%%%%%%%%%%%%%%%%%%%%%%%%%%%%%%%%%%%%%%%%%%%%%%%%%%%%%%%%%%%%%%%%%%

\section[Real data]{Test Case on real data}
%-------------------------------------------%

	\subsection{Data}
%%%%%%%%%%%%%%%%%%%%%%%%%%%%%%%%%%%%%%%%%%%%%%%%%%%%%%%%%%%%%%%%%%%%%%%%
\begin{frame}
%Test case in the South Atlantic area
\end{frame}
%%%%%%%%%%%%%%%%%%%%%%%%%%%%%%%%%%%%%%%%%%%%%%%%%%%%%%%%%%%%%%%%%%%%%%%%

	\subsection[Cost function]{Study of the cost function}
%%%%%%%%%%%%%%%%%%%%%%%%%%%%%%%%%%%%%%%%%%%%%%%%%%%%%%%%%%%%%%%%%%%%%%%%
\begin{frame}
%Evolution of the cost function for synthetic inversion, real data inversion
\end{frame}
%%%%%%%%%%%%%%%%%%%%%%%%%%%%%%%%%%%%%%%%%%%%%%%%%%%%%%%%%%%%%%%%%%%%%%%%  

	\subsection[Corrections]{Corrected velocity field}
%%%%%%%%%%%%%%%%%%%%%%%%%%%%%%%%%%%%%%%%%%%%%%%%%%%%%%%%%%%%%%%%%%%%%%%%
\begin{frame}
%Corrections on ssh and u,v
\end{frame}

\begin{frame}
%Lagrangian trajectories
\end{frame}
%%%%%%%%%%%%%%%%%%%%%%%%%%%%%%%%%%%%%%%%%%%%%%%%%%%%%%%%%%%%%%%%%%%%%%%%

%%%%%%%%%%%%%%%%%%%%%%%%%%%%%%%%%%%%%%%%%%%%%%%%%%%%%%%%%%%%%%%%%%%%%%%%
\begin{frame}
  \frametitle{Outline}
  \tableofcontents%[pausesections]
\end{frame}
%%%%%%%%%%%%%%%%%%%%%%%%%%%%%%%%%%%%%%%%%%%%%%%%%%%%%%%%%%%%%%%%%%%%%%%%


\section[Process model data]{Validation of the method on a process model}
%------------------------------------------------------------------------%

	\subsection{Model data}
%%%%%%%%%%%%%%%%%%%%%%%%%%%%%%%%%%%%%%%%%%%%%%%%%%%%%%%%%%%%%%%%%%%%%%%%
\begin{frame}
%Physicobiogeochemical Model
\end{frame}
%%%%%%%%%%%%%%%%%%%%%%%%%%%%%%%%%%%%%%%%%%%%%%%%%%%%%%%%%%%%%%%%%%%%%%%%

	\subsection{Inversion of a tracer image from the model}
%%%%%%%%%%%%%%%%%%%%%%%%%%%%%%%%%%%%%%%%%%%%%%%%%%%%%%%%%%%%%%%%%%%%%%%%
\begin{frame}
%Inversion with SST alone, Chl alone, SST and Chl, bidouille SST+Chl 
\end{frame} 
%%%%%%%%%%%%%%%%%%%%%%%%%%%%%%%%%%%%%%%%%%%%%%%%%%%%%%%%%%%%%%%%%%%%%%

%%%%%%%%%%%%%%%%%%%%%%%%%%%%%%%%%%%%%%%%%%%%%%%%%%%%%%%%%%%%%%%%%%%%%%%%
\begin{frame}
  \frametitle{Outline}
  \tableofcontents%[pausesections]
\end{frame}
%%%%%%%%%%%%%%%%%%%%%%%%%%%%%%%%%%%%%%%%%%%%%%%%%%%%%%%%%%%%%%%%%%%%%%%%

\section[Realistic Model data]{Limits of inversion using a realistic model}
%-------------------------------------------------------------------------%

	\subsection{Model data}
%%%%%%%%%%%%%%%%%%%%%%%%%%%%%%%%%%%%%%%%%%%%%%%%%%%%%%%%%%%%%%%%%%%%%%%%
\begin{frame}
%Salomon model
\end{frame}
%%%%%%%%%%%%%%%%%%%%%%%%%%%%%%%%%%%%%%%%%%%%%%%%%%%%%%%%%%%%%%%%%%%%%%%%

	\subsection{Similarities between FSLE and tracer}
%%%%%%%%%%%%%%%%%%%%%%%%%%%%%%%%%%%%%%%%%%%%%%%%%%%%%%%%%%%%%%%%%%%%%%%%
\begin{frame}
%Annual cycle
\end{frame}
%%%%%%%%%%%%%%%%%%%%%%%%%%%%%%%%%%%%%%%%%%%%%%%%%%%%%%%%%%%%%%%%%%%%%%%%

	\subsection{Inversion of a tracer image}
%%%%%%%%%%%%%%%%%%%%%%%%%%%%%%%%%%%%%%%%%%%%%%%%%%%%%%%%%%%%%%%%%%%%%%%%
\begin{frame}
%Feasability of the inversion 
\end{frame}
%%%%%%%%%%%%%%%%%%%%%%%%%%%%%%%%%%%%%%%%%%%%%%%%%%%%%%%%%%%%%%%%%%%%%%%%

\section[Conclusion]{Prospects and Conclusions}
%------------------------------------------------%

	\subsection{Prospects}
%%%%%%%%%%%%%%%%%%%%%%%%%%%%%%%%%%%%%%%%%%%%%%%%%%%%%%%%%%%%%%%%%%%%%%%%%
\begin{frame}
%Full DA
%SWOT Prospects
\end{frame}
%%%%%%%%%%%%%%%%%%%%%%%%%%%%%%%%%%%%%%%%%%%%%%%%%%%%%%%%%%%%%%%%%%%%%%%%%

	\subsection{Conclusions}
%%%%%%%%%%%%%%%%%%%%%%%%%%%%%%%%%%%%%%%%%%%%%%%%%%%%%%%%%%%%%%%%%%%%%%%%%
\begin{frame}
%Conclusions
\end{frame}
%%%%%%%%%%%%%%%%%%%%%%%%%%%%%%%%%%%%%%%%%%%%%%%%%%%%%%%%%%%%%%%%%%%%%%%%%

%%%%%%%%%%%%%%%%%%%%%%%%%%%%%%%%%%%%%%%%%%%%%%%%%%%%%%%%%%%%%%%%%%%%%%%%%
\begin{frame}
\begin{center}
Thank you for your attention 
\end{center}
\end{frame}
%%%%%%%%%%%%%%%%%%%%%%%%%%%%%%%%%%%%%%%%%%%%%%%%%%%%%%%%%%%%%%%%%%%%%%%%%
\end{document}
