%\documentclass[compress,notesonly]{beamer} %compress to make bars as small as possible
                                           %notesonly to add note not visible on screen (\note[]{})
%\documentclass[compress,slidescentered,notes=show]{beamer}
\documentclass[compress,slidescentered,notes=hide]{beamer}
\usepackage[latin1]{inputenc}
%\usepackage[T1]{fontenc}
\usepackage[english, french]{babel}
\usepackage{pifont}
\usepackage{beamerthemesplit}
\usepackage{multicol} %possibility to create columns
\usepackage{graphicx} %add pictures
\graphicspath{{./pict/}{pict_nemo/}{/data2/gaultier/NEMO/DIAG/}} %path to pictures
\usepackage{indentfirst}
\usepackage{tikz} %add schemes
\usetikzlibrary{shapes} %add diamonds shape to schemes
\usepackage{multimedia} %add video
\usepackage{subfigure}
%\usepackage[absolute,showboxes, overlay]{textpos}
%\textblockorigin{1mm}{1mm}
%\TPshowboxestrue % or false to display contour
\pdfpageattr {/Group << /S /Transparency /I true /CS /DeviceRGB>>}
\usenavigationsymbolstemplate{}
\usetheme{Darmstadt}
%\usetheme{Singapore}
\useoutertheme{smoothbars} %to add numerotation
\setbeamersize{text margin left=2mm, text margin right=2mm}
\usepackage{geometry} %put margin
\geometry{hmargin=0.25cm, vmargin=0.0cm}
\usepackage{color} %creation of own colors
%\usecolortheme[named=SeaGreen]{structure}
\definecolor{bleuclair}{rgb}{0.2,0.9,0.8}
%\definecolor{mycyan}{rgb}{.19,0.5,0.5}
\definecolor{mycyan}{rgb}{0.2,0.6,0.6}
\setbeamercolor*{palette primary}{use=structure,fg=white,bg=mycyan}
\setbeamercolor{block title}{bg=mycyan,fg=black}%bg=background, fg= foreground
\setbeamercolor{block body}{bg=lightgray,fg=black}%bg=background, fg= foreground
\setbeamercolor{structure}{bg=black, fg=mycyan}
\setbeamercolor{normal text}{fg=black}
\setbeamercolor{alerted text}{fg=red}
%\setbeamercolor{background canvas}{bg=white}
\setbeamercolor{frametitle}{fg=white}
\setbeamercolor{title}{fg=black}
\setbeamercolor{titlelike}{fg=black}
\setbeamercolor{title}{fg=black}
\setbeamercolor{section in sidebar}{fg=black}
\setbeamercolor{section in sidebar shaded}{fg= grey}
\setbeamercolor{subsection in sidebar}{fg=black}
\setbeamercolor{subsection in sidebar shaded}{fg= grey}
\setbeamercolor{itemize item}{fg=mycyan}
\setbeamercolor{section in tableofcontents}{fg=black,bg=black}
\setbeamercolor*{item projected}{fg = black, bg=mycyan} 
%\setbeamercolor{sidebar}{bg=red}
%\beamertemplatetransparentcovered %set transparancy
\useoutertheme{shadow}
\newcommand{\gu}[1]{#1}
\newcommand{\legende}[1]{\textit{\footnotesize #1}}
\newcommand{\ftitle}[1]{\begin{center}\textcolor{mycyan}{#1} \end{center}}
\renewcommand{\figurename}{Fig.}
%\renewcommand{\thesubfigure}{\alph{subfigure}}
%\setlength{\unitlength}{1cm} %to use pictures

%usepackage{default}

%\setbeamersize{text margin left=0cm}
%\setbeamersize{text margin right=0cm}
%\setbeamersize{text margin top=0cm}
%\setbeamersize{sidebar width left=0cm}
%\setbeamersize{sidebar width right=0cm}
%\usepackage{fullpage}
\setbeamertemplate{background}{\includegraphics[width=\paperwidth,height=\paperheight]{./pict/slide0006_background}}

\title{R\'eunion d'avancement du 15 mars}
%author[ ]{Lucile Gaultier, Jacques Verron, Pierre Brasseur, Jean-Michel Brankart}
\date{\textit{\today}}

%\logo{\includegraphics[height=1.5cm]{./pict/logo_meom.jpeg}}
%\logo{\insertframenumber/\inserttotalframenumber}
%\pgfdeclareimage[height=1.2cm]{legi}{./pict/logo_legi.jpeg}
%\logo{\pgfuseimage{legi}}

\begin{document}

%1%%%%%%%%%%%%%%%%%%%%%%%%%%%%%%%%%%%%%%%%%%%%%%%%%%%%%%%%%%%%%%%%%%%%%
\begin{frame}
  \maketitle
  \tableofcontents%[pausesections]

%\setbeamercolor*{palette primary}{use=structure,fg=white,bg=bleuclair}
%  \begin{center}
%    \includegraphics[height=1.5cm]{./pict/logo_meom.jpeg}
%    \hspace{0.5cm}
%    \includegraphics[height=1.5cm]{./pict/logo_legi.jpeg}
%    \hspace{0.5cm}
%    \includegraphics[height=1.5cm]{./pict/logo_cnrs.jpeg}
%    \hspace{0.5cm}
%    \includegraphics[height=1.5cm]{./pict/logo_cnes.jpeg}
%  \end{center}

%  \note{
% Le sujet de cette pr�sentation est l'inversion des informations sous-m�so�chelles contenues dans les images de traceurs afin de corriger des dynamiques m�so�chelles. 
% Plus pr�cis�ment, dans cette �tude, on utilise les informations contenues dans les observations spatiales de traceurs comme la Chlorophylle ou la temp�rature de surface de mer (SST) afin d'am�liorer l'estimation de la circulation oc�anique faite par les satellites altim�triques.
% Pour ce faire, on s'appuie sur une des sp�cialit�s de l'�quipe qui est l'utilisation des observations pour pr�dire et am�liorer les pr�dictions. 
%On utilise alors des techniques similaires � celles de l'assimilation de donn�es, les comp�tences du laboratoire sur ce sujet nous sont tr�s utiles. }
\end{frame}
%%%%%%%%%%%%%%%%%%%%%%%%%%%%%%%%%%%%%%%%%%%%%%%%%%%%%%%%%%%%%%%%%%%%%%%%

%\logo{\insertframenumber/\inserttotalframenumber}

\section{Les configurations test\'ees}
\subsection{Param\`etres du mod\`ele}
%2%%%%%%%%%%%%%%%%%%%%%%%%%%%%%%%%%%%%%%%%%%%%%%%%%%%%%%%%%%%%%%%%%%%%%
\begin{frame}
  \ftitle{Caract\'eristique du mod\`ele}
  \begin{itemize}
    \item Mod\`ele NEMO de type canal, coupl\'e avec LOBSTER pour la biog\'eochimie 
    \item For\c cage solaire en surface : 250 W.m$^{-2}$
    \item Sch\'ema de diffusion : op\'erateur bilaplacien
    \item Forcage en vent pour R6W et R2W : $u_{wind}$ = 0.1 cos($\phi$), $v_{wind}$ = 0  
  \end{itemize}
  \begin{center}
  \begin{tabular}{|c|c|c|}
   \hline
   & R6 / R6W & R2 / R2W \\
    \hline
    Domaine & $500 \times 500 \times 4$ km & $500 \times 500 \times 4$ km\\
            & $84 \times 84 \times 30$ pts &  $240 \times 252 \times 30$ pts\\ 
    \hline
    R\'esolution & 6km & 2km  \\
    \hline
    Diffusion & K=4.5 . $10^{10}$ & K=5 . $10^{09}$ \\
    \hline
  \end{tabular}
  \end{center}
\end{frame}
%%%%%%%%%%%%%%%%%%%%%%%%%%%%%%%%%%%%%%%%%%%%%%%%%%%%%%%%%%%%%%%%%%%%%%%%

\subsection{Initialisation}
%3%%%%%%%%%%%%%%%%%%%%%%%%%%%%%%%%%%%%%%%%%%%%%%%%%%%%%%%%%%%%%%%%%%%%%
\begin{frame}
%  \frametitle{Initialisation : dynamique}
\setcounter{subfigure}{0}
  \begin{figure}
    \subfigure[R6]{\label{a}\includegraphics[width=4.2cm]{R6_dyn_m01.png}}
    \subfigure[R6W]{\label{b}\includegraphics[width=4.2cm]{R6W_dyn_m01.png}}\\
    \subfigure[R2]{\label{c}\includegraphics[width=4.2cm]{R2BIG_dyn_m01.png}}
    \subfigure[R2W]{\label{d}\includegraphics[width=4.2cm]{R2W_dyn_m01.png}}
  \end{figure}
\end{frame}
%%%%%%%%%%%%%%%%%%%%%%%%%%%%%%%%%%%%%%%%%%%%%%%%%%%%%%%%%%%%%%%%%%%%%%%%

%4%%%%%%%%%%%%%%%%%%%%%%%%%%%%%%%%%%%%%%%%%%%%%%%%%%%%%%%%%%%%%%%%%%%%%
\begin{frame}
%  \frametitle{Initialisation : dynamique}
\setcounter{subfigure}{0}
  \begin{figure}
    \subfigure[R6]{\includegraphics[width=4cm]{R6_initial_sst.png}}
    \subfigure[R6W]{\includegraphics[width=4cm]{R6W_initial_sst.png}}\\
    \subfigure[R2]{\includegraphics[width=4cm]{R2BIG_initial_sst.png}}
    \subfigure[R2W]{\includegraphics[width=4cm]{R2W_initial_sst.png}}
  \end{figure}
(Mahadevan et al, 2010)
\end{frame}
%%%%%%%%%%%%%%%%%%%%%%%%%%%%%%%%%%%%%%%%%%%%%%%%%%%%%%%%%%%%%%%%%%%%%%%%

%5%%%%%%%%%%%%%%%%%%%%%%%%%%%%%%%%%%%%%%%%%%%%%%%%%%%%%%%%%%%%%%%%%%%%%
\begin{frame}
  \ftitle{Initialisation : biog\'eochimie}
  \begin{center}
    \includegraphics[width=0.8\linewidth]{R6_tra01.png}
%    \includegraphics[width=0.5\linewidth]{tra201.png}
  \end{center}
\end{frame}
%%%%%%%%%%%%%%%%%%%%%%%%%%%%%%%%%%%%%%%%%%%%%%%%%%%%%%%%%%%%%%%%%%%%%%%%


\section{Sorties du mod\`ele}
\subsection{Evolution des traceurs}
%6%%%%%%%%%%%%%%%%%%%%%%%%%%%%%%%%%%%%%%%%%%%%%%%%%%%%%%%%%%%%%%%%%%%%%
\begin{frame}
  \ftitle{R6} %Temp\'erature, Chlorophylle et Vorticit\'e potentielle}
  \begin{center}
    \movie[width=09cm,height=7.3cm,externalviewer]{\includegraphics[width=9cm]{R6_TCHL_m01_d01.png}}{./pict_nemo/R6_tra.avi}
  \end{center}
\end{frame}
%%%%%%%%%%%%%%%%%%%%%%%%%%%%%%%%%%%%%%%%%%%%%%%%%%%%%%%%%%%%%%%%%%%%%%%%
%7%%%%%%%%%%%%%%%%%%%%%%%%%%%%%%%%%%%%%%%%%%%%%%%%%%%%%%%%%%%%%%%%%%%%%
\begin{frame}
  \ftitle{R6W} %Temp\'erature, Chlorophylle et Vorticit\'e potentielle}
  \begin{center}
    \movie[width=09cm,height=7.3cm,externalviewer]{\includegraphics[width=9cm]{R6W_TCHL_m01_d01.png}}{./pict_nemo/R6W_tra.avi}
  \end{center}
\end{frame}
%%%%%%%%%%%%%%%%%%%%%%%%%%%%%%%%%%%%%%%%%%%%%%%%%%%%%%%%%%%%%%%%%%%%%%%%

%8%%%%%%%%%%%%%%%%%%%%%%%%%%%%%%%%%%%%%%%%%%%%%%%%%%%%%%%%%%%%%%%%%%%%%
\begin{frame}
  \ftitle{R2} %Temp\'erature, Chlorophylle et Vorticit\'e potentielle}
  \begin{center}
    \movie[width=09cm,height=7.3cm,externalviewer]{\includegraphics[width=9cm]{R2_TCHL_m01_d01.png}}{./pict_nemo/R2BIG_tra.avi}
  \end{center}
\end{frame}
%%%%%%%%%%%%%%%%%%%%%%%%%%%%%%%%%%%%%%%%%%%%%%%%%%%%%%%%%%%%%%%%%%%%%%%%

%8%%%%%%%%%%%%%%%%%%%%%%%%%%%%%%%%%%%%%%%%%%%%%%%%%%%%%%%%%%%%%%%%%%%%%
%\begin{frame}
%  \ftitle{R2W} %Temp\'erature, Chlorophylle et Vorticit\'e potentielle}
%  \begin{center}
%    \movie[width=09cm,height=7.3cm,externalviewer]{\includegraphics[width=9cm]{TCHL3_m01_d01.png}}{tra3EELR5.avi}
%  \end{center}
%\end{frame}
%%%%%%%%%%%%%%%%%%%%%%%%%%%%%%%%%%%%%%%%%%%%%%%%%%%%%%%%%%%%%%%%%%%%%%%%


\subsection{Energie cin\'etique}
%5%%%%%%%%%%%%%%%%%%%%%%%%%%%%%%%%%%%%%%%%%%%%%%%%%%%%%%%%%%%%%%%%%%%%%
\begin{frame}
  \ftitle{ Comportement Energ\'etique}
\setcounter{subfigure}{0}
  \begin{figure}
    \subfigure[R6]{\includegraphics[width=4.cm]{R6_tke_l0.png}}
    \subfigure[R6W]{\includegraphics[width=4.cm]{R6W_tke_l0.png}}\\ 
    \subfigure[R2]{\includegraphics[width=4.cm]{R2BIG_tke_l0.png}}
    \subfigure[R2W]{\includegraphics[width=4cm]{R2W_tke_l0.png}}
  \end{figure}
  \begin{block}{}
  L'aggrandissement du canal permet de maintenir les instabilit�s plus longtemps. \\    

  \end{block}
\end{frame}
%%%%%%%%%%%%%%%%%%%%%%%%%%%%%%%%%%%%%%%%%%%%%%%%%%%%%%%%%%%%%%%%%%%%%%%%

\subsection{FSLE}

%10%%%%%%%%%%%%%%%%%%%%%%%%%%%%%%%%%%%%%%%%%%%%%%%%%%%%%%%%%%%%%%%%%%%%
\begin{frame}
  \ftitle{FSLE pour R6}
  \setcounter{subfigure}{0}
  \begin{figure}
     \subfigure[Instationnaire]{\label{R6istat}\includegraphics[height=4cm]{EEL-R6_FSLESST_instat20_19690329.png}}
     \subfigure[Stationnaire]{\label{R6stat}\includegraphics[height=4cm]{EEL-R6_FSLESST_stat20_19690329.png}}
  \end{figure}
\end{frame}
%%%%%%%%%%%%%%%%%%%%%%%%%%%%%%%%%%%%%%%%%%%%%%%%%%%%%%%%%%%%%%%%%%%%%%%%

%10%%%%%%%%%%%%%%%%%%%%%%%%%%%%%%%%%%%%%%%%%%%%%%%%%%%%%%%%%%%%%%%%%%%%
\begin{frame}
  \ftitle{FSLE pour R6}
  \setcounter{subfigure}{0}
  \begin{figure}
     \subfigure[Instationnaire]{\label{R6istat}\includegraphics[height=4cm]{EEL-R6_FSLECHL_instat20_19690329.png}}
     \subfigure[Stationnaire]{\label{R6stat}\includegraphics[height=4cm]{EEL-R6_FSLECHL_stat20_19690329.png}}
  \end{figure}
\end{frame}
%%%%%%%%%%%%%%%%%%%%%%%%%%%%%%%%%%%%%%%%%%%%%%%%%%%%%%%%%%%%%%%%%%%%%%%%
%10%%%%%%%%%%%%%%%%%%%%%%%%%%%%%%%%%%%%%%%%%%%%%%%%%%%%%%%%%%%%%%%%%%%%
\begin{frame}
  \ftitle{FSLE pour R2}
  \setcounter{subfigure}{0}
  \begin{figure}
     \subfigure[Instationnaire]{\label{R2istat}\includegraphics[height=4cm]{EEL-R2BIG_FSLESST_instat20_19690429.png}}
     \subfigure[Stationnaire]{\label{R2stat}\includegraphics[height=4cm]{EEL-R2BIG_FSLESST_stat20_19690429.png}}
  \end{figure}
\end{frame}
%%%%%%%%%%%%%%%%%%%%%%%%%%%%%%%%%%%%%%%%%%%%%%%%%%%%%%%%%%%%%%%%%%%%%%%%

%10%%%%%%%%%%%%%%%%%%%%%%%%%%%%%%%%%%%%%%%%%%%%%%%%%%%%%%%%%%%%%%%%%%%%
\begin{frame}
  \ftitle{FSLE pour R2}
  \setcounter{subfigure}{0}
  \begin{figure}
     \subfigure[Instationnaire]{\label{R2istat}\includegraphics[height=4cm]{EEL-R2BIG_FSLESST_instat20_19690429.png}}
     \subfigure[Stationnaire]{\label{R2stat}\includegraphics[height=4cm]{EEL-R2BIG_FSLESST_stat20_19690429.png}}
  \end{figure}
\end{frame}
%%%%%%%%%%%%%%%%%%%%%%%%%%%%%%%%%%%%%%%%%%%%%%%%%%%%%%%%%%%%%%%%%%%%%%%%
\section{Plan d'exp\'erience}

%10%%%%%%%%%%%%%%%%%%%%%%%%%%%%%%%%%%%%%%%%%%%%%%%%%%%%%%%%%%%%%%%%%%%%
\begin{frame}
  \ftitle{Plan d'exp\'erience}
%  \movie[width=09cm,height=7cm,poster]{}{tra.avi}
  \begin{block}{Objectif 1}
    Comparer les FSLE \`a la structure du champ de SST et de Chlorophylle afin de mesurer le degr\'e de similarit\'e entre les FSLE issues de la vitesse vraie et les champs vrais de traceurs. 
  \end{block} 
  \begin{block}{Objectif 2}
    Corriger un champ de vitesse erron\'e l'aide de l'inversion des images de SST et de Chlorophylle
  \end{block}

\end{frame}
%%%%%%%%%%%%%%%%%%%%%%%%%%%%%%%%%%%%%%%%%%%%%%%%%%%%%%%%%%%%%%%%%%%%%%%%

\section{R\'eunion au LJK}
%3%%%%%%%%%%%%%%%%%%%%%%%%%%%%%%%%%%%%%%%%%%%%%%%%%%%%%%%%%%%%%%%%%%%%%
\begin{frame}
  \ftitle{M\'ethode de mesure de distance entre deux images}
%  \begin{center}
%    \includegraphics[width=0.5\linewidth]{dyn_m01.png}
%    \includegraphics[width=0.5\linewidth]{dyn2_m01.png}
%  \end{center}
\end{frame}
%%%%%%%%%%%%%%%%%%%%%%%%%%%%%%%%%%%%%%%%%%%%%%%%%%%%%%%%%%%%%%%%%%%%%%%%

\end{document}
