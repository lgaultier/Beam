%\documentclass[compress,notesonly]{beamer} %compress to make bars as small as possible
                                           %notesonly to add note not visible on screen (\note[]{})
%\documentclass[compress,slidescentered,notes=show]{beamer}
\documentclass[compress,slidescentered,notes=hide]{beamer}
\usepackage[latin1]{inputenc}
%\usepackage[T1]{fontenc}
\usepackage[english, french]{babel}
\usepackage{pifont}
\usepackage{beamerthemesplit}
\usepackage{multicol} %possibility to create columns
\usepackage{graphicx} %add pictures
\graphicspath{{./pict/}} %path to pictures
\usepackage{indentfirst}
\usepackage{tikz} %add schemes
\usetikzlibrary{shapes} %add diamonds shape to schemes
\usepackage{multimedia} %add video
%\usepackage[absolute,showboxes, overlay]{textpos}
%\textblockorigin{1mm}{1mm}
%\TPshowboxestrue % or false to display contour
\pdfpageattr {/Group << /S /Transparency /I true /CS /DeviceRGB>>}
\usenavigationsymbolstemplate{}
\usetheme{Warsaw}
%\useoutertheme{infolines} %to add numerotation
\usepackage{geometry} %put margin
\geometry{hmargin=0.25cm, vmargin=0.0cm}
\usepackage{color} %creation of own colors
%\usecolortheme[named=SeaGreen]{structure}
\definecolor{bleuclair}{rgb}{0.2,0.9,0.8}
%\definecolor{mycyan}{rgb}{.19,0.5,0.5}
\definecolor{mycyan}{rgb}{0.2,0.6,0.6}
\setbeamercolor*{palette primary}{use=structure,fg=white,bg=mycyan}
\setbeamercolor{block title}{bg=mycyan,fg=black}%bg=background, fg= foreground
\setbeamercolor{block body}{bg=lightgray,fg=black}%bg=background, fg= foreground
%\setbeamercolor{structure}{bg=black, fg=green}
\setbeamercolor{normal text}{fg=black}
\setbeamercolor{alerted text}{fg=red}
%\setbeamercolor{background canvas}{bg=white}
\setbeamercolor{frametitle}{fg=white}
\setbeamercolor{title}{fg=black}
\setbeamercolor{titlelike}{fg=black}
\setbeamercolor{title}{fg=black}
\setbeamercolor{section in sidebar}{fg=black}
\setbeamercolor{section in sidebar shaded}{fg= grey}
\setbeamercolor{subsection in sidebar}{fg=black}
\setbeamercolor{subsection in sidebar shaded}{fg= grey}
\setbeamercolor{itemize item}{fg=mycyan}
\setbeamercolor{section in tableofcontents}{fg=black,bg=black}
\setbeamercolor*{item projected}{fg = white, bg=mycyan} 
%\setbeamercolor{sidebar}{bg=red}
%\beamertemplatetransparentcovered %set transparancy
\useoutertheme{shadow}
\newcommand{\gu}[1]{#1}
\newcommand{\legende}[1]{\textit{\footnotesize #1}}
\renewcommand{\figurename}{Fig.}
\setlength{\unitlength}{1cm} %to use pictures
%usepackage{default}
%\setbeamersize{text margin left=0cm}
%\setbeamersize{text margin right=0cm}
%\setbeamersize{text margin top=0cm}
%\setbeamersize{sidebar width left=0cm}
%\setbeamersize{sidebar width right=0cm}
%\usepackage{fullpage}
\setbeamertemplate{background}{\includegraphics[width=\paperwidth,height=\paperheight]{./pict/slide0006_background}}

\title{Modele coupl� physique-biog�ochimie, de type canal, de r�solution 6km}
%author[ ]{Lucile Gaultier, Jacques Verron, Pierre Brasseur, Jean-Michel Brankart}
\date{\textit{\today}}

%\logo{\includegraphics[height=1.5cm]{./pict/logo_meom.jpeg}}
%\logo{\insertframenumber/\inserttotalframenumber}
%\pgfdeclareimage[height=1.2cm]{legi}{./pict/logo_legi.jpeg}
%\logo{\pgfuseimage{legi}}

\begin{document}

%1%%%%%%%%%%%%%%%%%%%%%%%%%%%%%%%%%%%%%%%%%%%%%%%%%%%%%%%%%%%%%%%%%%%%%
\begin{frame}
  \maketitle
  \tableofcontents%[pausesections]

%\setbeamercolor*{palette primary}{use=structure,fg=white,bg=bleuclair}
%  \begin{center}
%    \includegraphics[height=1.5cm]{./pict/logo_meom.jpeg}
%    \hspace{0.5cm}
%    \includegraphics[height=1.5cm]{./pict/logo_legi.jpeg}
%    \hspace{0.5cm}
%    \includegraphics[height=1.5cm]{./pict/logo_cnrs.jpeg}
%    \hspace{0.5cm}
%    \includegraphics[height=1.5cm]{./pict/logo_cnes.jpeg}
%  \end{center}

%  \note{
% Le sujet de cette pr�sentation est l'inversion des informations sous-m�so�chelles contenues dans les images de traceurs afin de corriger des dynamiques m�so�chelles. 
% Plus pr�cis�ment, dans cette �tude, on utilise les informations contenues dans les observations spatiales de traceurs comme la Chlorophylle ou la temp�rature de surface de mer (SST) afin d'am�liorer l'estimation de la circulation oc�anique faite par les satellites altim�triques.
% Pour ce faire, on s'appuie sur une des sp�cialit�s de l'�quipe qui est l'utilisation des observations pour pr�dire et am�liorer les pr�dictions. 
%On utilise alors des techniques similaires � celles de l'assimilation de donn�es, les comp�tences du laboratoire sur ce sujet nous sont tr�s utiles. }
\end{frame}
%%%%%%%%%%%%%%%%%%%%%%%%%%%%%%%%%%%%%%%%%%%%%%%%%%%%%%%%%%%%%%%%%%%%%%%%

%\logo{\insertframenumber/\inserttotalframenumber}
\section{Description du Mod�le}
%1%%%%%%%%%%%%%%%%%%%%%%%%%%%%%%%%%%%%%%%%%%%%%%%%%%%%%%%%%%%%%%%%%%%%%
\begin{frame}
  \frametitle{Caract�ristique du mod�le}
  \begin{itemize}
    \item Mod�le NEMO de type canal, coupl� avec LOBSTER pour la biog�ochimie 
    \item Domaine : $160 \times 500 \times 4$ km ($28 \times 84 \times 30$ points de grilles)
    \item R�solution : 6 km 
    \item Structures sous-m�so/m�so�chelles g�n�r�es par un jet barocline instable. 

  \end{itemize}

\end{frame}
%%%%%%%%%%%%%%%%%%%%%%%%%%%%%%%%%%%%%%%%%%%%%%%%%%%%%%%%%%%%%%%%%%%%%%%%

%1%%%%%%%%%%%%%%%%%%%%%%%%%%%%%%%%%%%%%%%%%%%%%%%%%%%%%%%%%%%%%%%%%%%%%
\begin{frame}
  \frametitle{Initialisation : dynamique}
  \begin{center}
    \includegraphics[width=8cm]{dyn_m01.png}\\
  \end{center}
\end{frame}
%%%%%%%%%%%%%%%%%%%%%%%%%%%%%%%%%%%%%%%%%%%%%%%%%%%%%%%%%%%%%%%%%%%%%%%%

%1%%%%%%%%%%%%%%%%%%%%%%%%%%%%%%%%%%%%%%%%%%%%%%%%%%%%%%%%%%%%%%%%%%%%%
\begin{frame}
  \frametitle{Initialisation : dynamique}
  \begin{center}
    \includegraphics[width=8cm]{initial_sst.png}\\
  \end{center}
\end{frame}
%%%%%%%%%%%%%%%%%%%%%%%%%%%%%%%%%%%%%%%%%%%%%%%%%%%%%%%%%%%%%%%%%%%%%%%%

%1%%%%%%%%%%%%%%%%%%%%%%%%%%%%%%%%%%%%%%%%%%%%%%%%%%%%%%%%%%%%%%%%%%%%%
\begin{frame}
  \frametitle{Initialisation : biog�ochimie}
  \begin{center}
    \includegraphics[width=8cm]{tra01.png}\\
  \end{center}
\end{frame}
%%%%%%%%%%%%%%%%%%%%%%%%%%%%%%%%%%%%%%%%%%%%%%%%%%%%%%%%%%%%%%%%%%%%%%%%


\section{Sorties du mod�le}

%1%%%%%%%%%%%%%%%%%%%%%%%%%%%%%%%%%%%%%%%%%%%%%%%%%%%%%%%%%%%%%%%%%%%%%
\begin{frame}
  \frametitle{Temp�rature, Chlorophylle et Vorticit� potentielle}
  \begin{center}
    \movie[width=09cm,height=7.3cm,externalviewer]{\includegraphics[width=9cm]{TCHL_m01_d01.png}}{tra.avi}
  \end{center}
\end{frame}
%%%%%%%%%%%%%%%%%%%%%%%%%%%%%%%%%%%%%%%%%%%%%%%%%%%%%%%%%%%%%%%%%%%%%%%%

\section{Plan d'exp�rience}

%1%%%%%%%%%%%%%%%%%%%%%%%%%%%%%%%%%%%%%%%%%%%%%%%%%%%%%%%%%%%%%%%%%%%%%
\begin{frame}
  \frametitle{Plan d'exp�rience}
%  \movie[width=09cm,height=7cm,poster]{}{tra.avi}
  \begin{block}{Objectif 1}
    Comparer les FSLE � la structure du champ de SST et de Chlorophylle afin de mesurer le degr� de similarit� entre les FSLE issues de la vitesse vraie et les champs vrais de traceurs. 
  \end{block} 
  \begin{block}{Objectif 2}
    Corriger un champ de vitesse erron� � l'aide de l'inversion des images de SST et de Chlorophylle
  \end{block}

\end{frame}
%%%%%%%%%%%%%%%%%%%%%%%%%%%%%%%%%%%%%%%%%%%%%%%%%%%%%%%%%%%%%%%%%%%%%%%%

\end{document}
