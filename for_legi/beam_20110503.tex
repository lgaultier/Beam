%\documentclass[compress,notesonly]{beamer} %compress to make bars as small as possible
                                           %notesonly to add note not visible on screen (\note[]{})
\documentclass[compress,slidescentered,notes=hide]{beamer}
\usepackage[latin1]{inputenc}
%\usepackage[T1]{fontenc}
\usepackage[english]{babel}

\usepackage{beamerthemesplit}
\usepackage{multicol} %possibility to create columns
\usepackage{graphicx} %add pictures
\graphicspath{{./pict/}} %path to pictures
\usepackage{indentfirst}
\usepackage{tikz} %add schemes
\usetikzlibrary{shapes} %add diamonds shape to schemes
\usepackage{multimedia} %add video
%\usepackage[absolute,showboxes, overlay]{textpos}
%\textblockorigin{0.5,0.5}
%\TPshowboxestrue % or false to display contour

\usenavigationsymbolstemplate{}
\usetheme{Warsaw}
%\useoutertheme{infolines} %to add numerotation
\usepackage{geometry} %put margin
\geometry{hmargin=0.25cm, vmargin=0.0cm}
\usepackage{color} %creation of own colors
%\usecolortheme[named=SeaGreen]{structure}
\definecolor{bleuclair}{rgb}{0.7,0.7,0.4}
%\definecolor{mycyan}{rgb}{.19,0.5,0.5}
\definecolor{mycyan}{rgb}{0.2,0.6,0.6}
\setbeamercolor*{palette primary}{use=structure,fg=white,bg=mycyan}
\setbeamercolor{block title}{bg=mycyan,fg=black}%bg=background, fg= foreground
\setbeamercolor{block body}{bg=lightgray,fg=black}%bg=background, fg= foreground
%\setbeamercolor{structure}{bg=black, fg=green}
\setbeamercolor{normal text}{fg=black}
\setbeamercolor{alerted text}{fg=red}
%\setbeamercolor{background canvas}{bg=white}
\setbeamercolor{frametitle}{fg=white}
\setbeamercolor{title}{fg=black}
\setbeamercolor{titlelike}{fg=black}
\setbeamercolor{title}{fg=black}
\setbeamercolor{section in sidebar}{fg=black}
\setbeamercolor{section in sidebar shaded}{fg= grey}
\setbeamercolor{subsection in sidebar}{fg=black}
\setbeamercolor{subsection in sidebar shaded}{fg= grey}
\setbeamercolor{itemize item}{fg=mycyan}
\setbeamercolor{section in tableofcontents}{fg=black,bg=black}
\setbeamercolor*{item projected}{fg = white, bg=mycyan} 
%\setbeamercolor{sidebar}{bg=red}

\useoutertheme{shadow}
\newcommand{\gu}[1]{#1}
\newcommand{\legende}[1]{\textit{\footnotesize #1}}
\renewcommand{\figurename}{Fig.}
%usepackage{default}
%\setbeamersize{text margin left=0cm}
%\setbeamersize{text margin right=0cm}
%\setbeamersize{text margin top=0cm}
%\setbeamersize{sidebar width left=0cm}
%\setbeamersize{sidebar width right=0cm}
%\usepackage{fullpage}
\setbeamertemplate{background}{\includegraphics[width=\paperwidth,height=\paperheight]{./pict/slide0006_background}}

\title{On the inversion of sub-mesoscale information \hspace{4cm} to correct mesoscale velocity}
\author[Liege Colloquium]{Lucile Gaultier, Jacques Verron, Pierre Brasseur, Jean-Michel Brankart}
\date{\textit{\today}}

%\logo{\includegraphics[height=1.5cm]{./pict/logo_meom.jpeg}}
%\logo{\insertframenumber/\inserttotalframenumber}
%\pgfdeclareimage[height=1.2cm]{legi}{./pict/logo_legi.jpeg}
%\logo{\pgfuseimage{legi}}

\begin{document}

%1%%%%%%%%%%%%%%%%%%%%%%%%%%%%%%%%%%%%%%%%%%%%%%%%%%%%%%%%%%%%%%%%%%%%%
\begin{frame}
  \maketitle

  \begin{center}
    \includegraphics[height=1.5cm]{./pict/logo_meom.jpeg}
    \hspace{0.5cm}
    \includegraphics[height=1.5cm]{./pict/logo_legi.jpeg}
    \hspace{0.5cm}
    \includegraphics[height=1.5cm]{./pict/logo_cnrs.jpeg}
    \hspace{0.5cm}
    \includegraphics[height=1.5cm]{./pict/logo_cnes.jpeg}
  \end{center}
  \note{Satellites can observe geostrophic velocity fields, at scales higher than mesoscales whereas we can easily find tracers data at small scales. MERIS can provide data at a resolution as low as 200m. The idea of this study is to use thi high resolution information to control dynamics at larger scales. }
\end{frame}
%%%%%%%%%%%%%%%%%%%%%%%%%%%%%%%%%%%%%%%%%%%%%%%%%%%%%%%%%%%%%%%%%%%%%%%%


\logo{\insertframenumber/\inserttotalframenumber}

%SEVERAL SCALES IN THE OCEAN
%2%%%%%%%%%%%%%%%%%%%%%%%%%%%%%%%%%%%%%%%%%%%%%%%%%%%%%%%%%%%%%%%%%%%%%%
\begin{frame}[plain]
  \centering
  \frametitle{\centering Mesoscale dynamics cascade into sub-mesoscale dynamics}
  \begin{figure}
    \includegraphics[width=0.7\linewidth]{scales.jpg}
  \end{figure}
  \note{ The scales in which we are interested are located between 1km and 100 km. Mesoscale dynamics cascade into sub-mesoscale dunamics. At sub-mesoscales there is a strong interaction between the physics and the biogeochemistry sot that the sub-mesoscale dynamic impact tracer fields}
\end{frame}
%ckward (red) and forward (green) integration of trajectories}
%%%%%%%%%%%%%%%%%%%%%%%%%%%%%%%%%%%%%%%%%%%%%%%%%%%%%%%%%%%%%%%%%%%%%%%%


%3%%%%%%%%%%%%%%%%%%%%%%%%%%%%%%%%%%%%%%%%%%%%%%%%%%%%%%%%%%%%%%%%%%%%%%
\begin{frame}
  \frametitle{Context of this study}
  $\bullet$ Lyapunov exponents can be seen as a proxy for tracers (d'Ovidio \& al (2004), Lehahn\& al (2007)): 
  \begin{tikzpicture}
  \node[color=blue, text width=3cm, text centered] (UV) at (0,0) {Mesoscale velocity}; 
  \node[color=green, text width=3cm, text centered] (FLE) at (4.7,0) {Lyapunov exponents}; 
  \node[color=green, text width=3cm, text centered] (TRA) at (9,0) {Sub-mesoscale image tracer};

  \draw[->,>=latex] (1.7,0.2)--(3.7,0.2);
  \draw[->,>=latex] (5.7,0.2)--(7.7,0.2);
  \visible<2->{\draw[->,>=latex,color=red] (3.7,-0.2)--(1.7,-0.2);
               \draw[->,>=latex,color=red] (7.7,-0.2)--(5.7,-0.2);}
  
  \end{tikzpicture}
  \begin{figure}
    \includegraphics[width=0.3\linewidth]{aviso_19904_med_.png}
    \hspace{0.1cm}
    \includegraphics[width=0.33\linewidth]{fsle_48_stat_reg_19904_med.jpg}
    \hspace{0.1cm}
    \includegraphics[width=0.33\linewidth]{pict/A2004184123500_L2_LAC_SST.jpg}
    \hspace{0.1cm} 
  \end{figure}
  \vspace{0.3cm}
  \visible<2->{\alert{$\bullet$} Inversion of sub-mesoscale tracer information to correct mesoscale velocity}

  \note{Lately, Some Lagrangian tools such as Lyapunov exponents have demonstrated to be a good proxy of the sub-mesoscale stirring in the ocean. The stretching lines and the local stirring can be derived using altimetric mesoscale velocity. The filaments are very similar to the frontal structure found in tracers as you can see on the sea surface temperature. The goal of this study is to prove that the inversion is feasible ; from a submesoscale image tracer, we want to correct the mesoscale velocity using Lyapunov Exponent as a proxy. We must prove that Lyapunov exponents are invertible into larger scales velocity before doing the full inversion. }
\end{frame}
%%%%%%%%%%%%%%%%%%%%%%%%%%%%%%%%%%%%%%%%%%%%%%%%%%%%%%%%%%%%%%%%%%%%%%%%

%4%%%%%%%%%%%%%%%%%%%%%%%%%%%%%%%%%%%%%%%%%%%%%%%%%%%%%%%%%%%%%%%%%%%%%%
\begin{frame}
  \frametitle{Outline}
  \tableofcontents%[pausesections]
\end{frame}
%%%%%%%%%%%%%%%%%%%%%%%%%%%%%%%%%%%%%%%%%%%%%%%%%%%%%%%%%%%%%%%%%%%%%%%%

	\section{Lyapunov exponents}

%5%%%%%%%%%%%%%%%%%%%%%%%%%%%%%%%%%%%%%%%%%%%%%%%%%%%%%%%%%%%%%%%%%%%%%%
\begin{frame}
  \frametitle{Physical meaning of Lyapunov Exponents}
  Lyapunov exponents are defined as the exponential rate of separation, averaged over time \\ 
  \vspace{1cm}
  \centering
%insert graph of advecting trajectories to get stable or unstable manifolds
%to I need to specify backwards = Stable = cf Lehahn2007
  \begin{columns}
    \begin{column}{0.5\textwidth}
%      \begin{figure}
%        \includegraphics[width=1\linewidth]{manifold.png}
%      \end{figure}
       \begin{center}
       \begin{tikzpicture}
%    \node (W) at (-2,0) {};
%    \node (E) at (2,0.6) {};
%    \node (C) at (0,0) {H};
%    \node (N) at (-0.6,2) {};
%    \node (S) at (0.6,-2) {};

%    \draw[->>,>=latex] (W) edge[bend right] (C);
%    \draw[->>,>=latex] (E) edge[bend right] (C);
%    \draw[->>,>=latex] (C) edge[bend left] (N); 
%    \draw[->>,>=latex] (C) edge[bend left] (S);
        \node (xi) at (0,0.2) {Xi};
        \node (xf) at (4,1) {Xf};
        \node (yi) at (0,-0.2) {Yi};
        \node (yf) at (4,-0.8) {Yf};
        \node (di) at (0.82,0) {$\delta_{initial}$};
        \node (df) at (4.15,0.1) {$\delta_{final}$};  

        \draw[-,>=latex] (xf) to[bend left=10] (xi);
        \draw[-,>=latex] (yf) edge[bend right=10] (yi);
        \draw[<->,>=latex] (0.3,0.2)--(0.3,-0.2);
        \draw[<->,>=latex] (3.7,0.95)--(3.7,-0.75);
      \end{tikzpicture}
      \end{center}
    \end{column}
    \begin{column}{0.49\textwidth}
      \begin{block}{}
      $$\lambda  = \frac{1}{T} \times log(\frac{\delta_{final}}{\delta_{initial}}) $$
      \end{block}
%      \vspace{1cm}
%      \legende{Backward (red) and forward (green) integration of trajectories}
    \end{column}
  \end{columns}
 \vspace{1cm} 
 Lyapunov exponents constitute Lagrangian transport barriers between different regions (Lehahn \& al (2007)). 
%  Stable and Unstable Manifolds constitute Lagrangian transport barriers between different regions because they are material invariant curves that cannot be crossed by purely advective process
  \note{Lyapunov exponents are defined as the exponential rate of separation, averaged over time. In this study we'll use Finite time Lyapunov Exponent. In practical terms we integrate the trajectories of particles and calculate th final distance of two particles, initially at distance deltai after a time T. }
\end{frame}
%%%%%%%%%%%%%%%%%%%%%%%%%%%%%%%%%%%%%%%%%%%%%%%%%%%%%%%%%%%%%%%%%%%%%%%%

%6%%%%%%%%%%%%%%%%%%%%%%%%%%%%%%%%%%%%%%%%%%%%%%%%%%%%%%%%%%%%%%%%%%%%%%
\begin{frame}
  \frametitle{Are Lyapunov exponents a reliable proxy/image?}
  \begin{columns}
    \begin{column}{0.5\textwidth}
      \begin{figure}%[!htbp]
        \includegraphics[width=6cm]{pict/fsle_48_stat_reg_19904_med.jpg}\\
        \legende{FSLE, June 30, 2004}
      \end{figure}
    \end{column}
    \begin{column}{0.5\textwidth}
      \begin{figure} 
        \includegraphics[width=6cm]{pict/A2004184123500_L2_LAC_SST.jpg}\\
        \legende{Sea Surface Temperature, July 03, 2004}
      \end{figure}
    \end{column}
  \end{columns}
  \vspace{1cm}
  Maximum lines of Lyapunov exponents and frontal tracer structures present similar patterns (d'Ovidio \& al (2004)).
  \note{If we compare Lyapunov exponents image with tracer image, we can find a similar pattern between the maximum lines of Lyapunov exponents and frontal tracer structure (that is to say the gradiant). Both are Lagrangian tracer barriers.}
\end{frame}
%%%%%%%%%%%%%%%%%%%%%%%%%%%%%%%%%%%%%%%%%%%%%%%%%%%%%%%%%%%%%%%%%%%%%%%%

	\section[Methodology]{Methodology of the inversion}

%7%%%%%%%%%%%%%%%%%%%%%%%%%%%%%%%%%%%%%%%%%%%%%%%%%%%%%%%%%%%%%%%%%%%%%%
\begin{frame}
  \centering
  \frametitle{An exploratory study}
  \begin{block}{}
    \begin{itemize}
      \item \textbf{Step 1}: Inversion of synthetic sub-mesoscale images to larger scale ocean circulation
        (twin experiment approach)
      \item \textbf{Step 2}: Inversion of sub-mesoscale ocean color or sea surface temperature to larger scale ocean circulation
    \end{itemize}
  \end{block} 
  \note{The goal of our study is to invert submesoscale information from tracer to correct mesoscale velocity. 
The first requirement is that Lyapunov exponent are invertibles , after checking that this is the case, we can invert sub-mesoscale tracer information to larger ocean circulation.}
\end{frame}
%%%%%%%%%%%%%%%%%%%%%%%%%%%%%%%%%%%%%%%%%%%%%%%%%%%%%%%%%%%%%%%%%%%%%%%

%%9%%%%%%%%%%%%%%%%%%%%%%%%%%%%%%%%%%%%%%%%%%%%%%%%%%%%%%%%%%%%%%%%%%%%%%
%\begin{frame}
%  \frametitle{Methodology}
%   $\bullet$ Velocity panel using Principal Compenent Analyes with all velocity field available
% $$u = \bar{u}+ \sum_{i=0}^n{x(i)\delta u(i)}$$
%  number of degrees of freedom reduced using only 100 or less EOFs. \\

%%Mathematiques equations defining space and sub space
%% Non linéarité of the pb, fonction cout complexe,
 %  $\bullet$ Assumption of the gaussianity of the velocity error panel \\
%   $\bullet$ The Cost function: the distance between the model and the observation
%   $$J(u)=\|\lambda(u)- \lambda_{obs}\| + background\ term $$
%  Minimization of this cost function complex because of many local minima
%  \only<2>{\includegraphics[width=0.7\linewidth]{Jiter_record6_HR.png}}
%\end{frame}
%%%%%%%%%%%%%%%%%%%%%%%%%%%%%%%%%%%%%%%%%%%%%%%%%%%%%%%%%%%%%%%%%%%%%%%%

%9%%%%%%%%%%%%%%%%%%%%%%%%%%%%%%%%%%%%%%%%%%%%%%%%%%%%%%%%%%%%%%%%%%%%%%
\begin{frame}
  \frametitle{Methodology}
%\temporal<n> {before}{at n}{after}
  \alt<2>{\centering\includegraphics[width=0.5\linewidth]{J2_med19904.png}\includegraphics[width=0.5\linewidth]{Jiter_record6_HR.png}\\}
  {
  $\bullet$ Velocity panel using Principal Component Analysis with all velocity fields available
  $$\textbf{u}_k = \bar{\textbf{u}} + \sum_{i=0}^n{\underbrace{a_k^i}_{Eigenvalue}\underbrace{\textbf{u}^i}_{EOF_{}}}$$
  The number of degrees of freedom is reduced using only 100 or less EOFs. \\
  \vspace{0.2cm} 
%Mathematiques equations defining space and sub space
% Non linéarité of the pb, fonction cout complexe,
  $\bullet$ Assumption of the gaussianity of the velocity error panel: 
  The velocity errors are normally distributed with 0 mean and covariance \textbf{P} (the covariance of the time sequence) $\delta\textbf{u} \simeq \mathcal{N}(0,\textbf{P})$ \\ 
 % {{\includegraphics[width=0.7\linewidth]{Jiter_record6_HR.png}}
  \vspace{0.2cm}
  $\bullet$ Integration of trajectories to derive Lyapunov exponents: $\lambda  = \frac{1}{T} \times log(\frac{\delta_{f}}{\delta_{i}}) $ \\ } 
  $\bullet$ The Cost function is the distance between the model and the observation
  $$J(u)=\|\lambda(u)- \lambda_{obs}\| + background\ term $$
  Minimization of this cost function complex because of many local minima
  \note{The first step is to build a velocity error panel using PCA. 
We have to do the PCA using all data available on the chosen area, which mean we have a space of EOF as huge as the number of velocity field. 
This number can be reduced using the principal components. I decided to keep at least 95\% of the variability. 
Using the EOF we can build the velocity error panel, assuming that the distribution of errors is gaussian.
As the perturbed velocity is created we can derive the corresponding Lyapunov exponents image. 
The cost fonction is the distance between the Lyapunov exponents image and the observation. 
The minimization of this cost function is quite complex because of the strong non-linearity of the problem and the number of local minima. 
}
\end{frame}
%%%%%%%%%%%%%%%%%%%%%%%%%%%%%%%%%%%%%%%%%%%%%%%%%%%%%%%%%%%%%%%%%%%%%%%%

%8%%%%%%%%%%%%%%%%%%%%%%%%%%%%%%%%%%%%%%%%%%%%%%%%%%%%%%%%%%%%%%%%%%%%%%
\begin{frame}
  \frametitle{Inversion algorithm}
  \begin{tikzpicture}
%\draw (0,0) circle (1) ;
    \node[draw, color=blue] (UV) at (0,4) {Mesoscale Velocity};
    \node[draw] (pict) at (0,2.3){\includegraphics[width=2.8cm]{aviso_19904_med_.png}};
    \node[draw, color=green] (SST) at (0,-2) {Tracer Image};
    \node[draw] (pict2) at (0,0){\includegraphics[width=2.8cm]{A2004184123500_L2_LAC_SST.jpg}};
    \node[draw, color=blue] (CUV) at (9,1) {Corrected Velocity};
    \node[draw] (pict5) at (9,2.5){\includegraphics[width=2.8cm]{aviso_velmap19904_med37133_ch2.png}};
    \node[draw] (pict3) at (5,3){\includegraphics[width=2.8cm]{pfsle_48_stat_reg_19904_med.jpg}};
    \node[draw] (pict4) at (5,-1){\includegraphics[width=2.8cm]{A2004184123500_L2_LAC_SST_record_bin6.jpg}};
    \node[draw] (Os) at (5,1) {Osmium};
    \draw[->,>=latex] (UV) -- (Os);
    \draw[->,>=latex] (SST) -- (Os);
    \draw[->,>=latex] (Os) -- (CUV);
  \end{tikzpicture}
%\line
%ad pictures fsle tracer velocity 
  \note{The cost function is the difference between Lyapunov Exponents derived from a mesoscale velocity fields and the norm of the gradient of the tracers. We try to find the mesoscale velocity which minimize the gap between those two items. In the case of o twin experiment, to invert Lyapunov exponents, the tracer image is the Lyapunov exponents of the true mesoscale velocity field}   
\end{frame}
%%%%%%%%%%%%%%%%%%%%%%%%%%%%%%%%%%%%%%%%%%%%%%%%%%%%%%%%%%%%%%%%%%%%%%%%

	\section{Test Case}

%10%%%%%%%%%%%%%%%%%%%%%%%%%%%%%%%%%%%%%%%%%%%%%%%%%%%%%%%%%%%%%%%%%%%%%
\begin{frame}
  \centering
  \frametitle{Choice of a Study area}
  \begin{block}{Required by Lyapunov exponent}
    \begin{itemize}
      \item Being far from any coast : Problem with particules advection in the presence of land
      \item Being far from any upwelling or downwelling : Vertical velocity is not taken into account in the calculation of Lyapunov exponent
    \end{itemize}
  \end{block}
  \begin{block}{Required by tracer}
    \begin{itemize}
      \item Presence of heterogeneity in the tracer to detect filament
      \item Low cloud cover: Visible and Near IR wavelength do not go through clouds
      \item Presence of an unstable manifold
    \end{itemize}
  \end{block}
  \note{The choice of the study area is an important step...  }
\end{frame}
%%%%%%%%%%%%%%%%%%%%%%%%%%%%%%%%%%%%%%%%%%%%%%%%%%%%%%%%%%%%%%%%%%%%%%%%

%11%%%%%%%%%%%%%%%%%%%%%%%%%%%%%%%%%%%%%%%%%%%%%%%%%%%%%%%%%%%%%%%%%%%%
\begin{frame}
  \centering
  \frametitle{Test case : Mediterranean Sea}
  \begin{itemize}
    \item \textbf{Region}: Mediterranean Sea, from $4.8^oE$ to $8^oE$, from $38.2^oN$ to $40.^oN$
    \item \textbf{Time Range}: from 1998 to June 2009, 595 velocity maps
    \item \textbf{Velocity fields}: AVISO altimeter data
    \item \textbf{Resolution}: $1/8^o$, grid points: 26*17
    \item \textbf{FSLE Resolution}: $1/48^o$, grid points: 119*86
  \end{itemize}
%  \vsapce{1cm}
  \begin{itemize}
    \item \textbf{SST field}: Data from MODIS captor, L2 product
    \item \textbf{Resolution needed to detect filament}: $1/100^o$
    %resolution needed to detect filament $1/100^o$, to match fsle $1/50^o$
  \end{itemize}
  \note{The area chosen for this test is the Western Mediterranean Sea, north of the African current. }
\end{frame}
%%%%%%%%%%%%%%%%%%%%%%%%%%%%%%%%%%%%%%%%%%%%%%%%%%%%%%%%%%%%%%%%%%%%%%%%

%12%%%%%%%%%%%%%%%%%%%%%%%%%%%%%%%%%%%%%%%%%%%%%%%%%%%%%%%%%%%%%%%%%%%%%
\begin{frame}
  \frametitle{Inversion algorithm}
  \begin{tikzpicture}
    \node[draw, color=blue,text width=3cm, text centered] (UV) at (0,0) {velocity = background+pert};
    \node[draw, color=green, text width=3cm, text centered] (FLE) at (4,0) {Lyapunov Exponents};
    \node[draw, color=green, text width = 3cm, text centered] (SST) at (4,-1) {Tracer structure};
    \node[draw, text width=2cm, text centered] (J) at (7.3,-0.5) {Cost function};
    \node[text width=2cm] (move) at (10,-0.5) {Is move accepted?};
    \node[draw,diamond, aspect=2, fill=gray] (no) at (5,2) {\tiny{NO}};
    \node[draw, diamond, aspect=2, fill=gray] (yes) at (7,-3) {\tiny{YES}};
    \node[draw, color=blue, text width=3cm, text centered] (nUV) at (4,-3) {background = velocity};
    
    \draw[->,>=latex] (move) to[bend right=15] (no);
    \draw[->,>=latex] (move) to[bend left=15] (yes);
    \draw[->,>=latex] (UV) -- (FLE);
    \draw[->,>=latex] (FLE) -- (J);
    \draw[->,>=latex] (SST) -- (J);
    \draw[->,>=latex] (J) -- (move);
    \draw[->,>=latex] (no) to[bend right=15] (UV);
    \draw[->,>=latex] (nUV) to[bend left=15] (UV);
   \draw[->,>=latex] (yes) -- (nUV);
  \end{tikzpicture}
%  $\bullet$ Cost function: $J = \|\hat{\lambda}_{fsle}-\hat{\lambda}_{tracer}\| \times (1+log\left(\frac{\|u\|}{\|u_{aviso}\|} \right))$\\ 
  \note{After building the velocity error panel, using 50 eofs, we can compute the perturbation and correct the mesoscale velocity field}
\end{frame}
%%%%%%%%%%%%%%%%%%%%%%%%%%%%%%%%%%%%%%%%%%%%%%%%%%%%%%%%%%%%%%%%%%%%%%%%

%13%%%%%%%%%%%%%%%%%%%%%%%%%%%%%%%%%%%%%%%%%%%%%%%%%%%%%%%%%%%%%%%%%%%%%
\begin{frame}
  \frametitle{Results}
  \begin{columns}
    \begin{column}{0.5\textwidth}
      \begin{figure}%[!htbp]
        \includegraphics[width=5.9cm]{./pict/aviso_velmap19904_med_osmium.png}\\
        \legende{Aviso velocity, June 30, 2004, \\cost function: 0.33}
      \end{figure}
    \end{column}
    \begin{column}{0.5\textwidth}
      \begin{figure}
        \includegraphics[width=5.9cm]{./pict/aviso_velmap19904_med_osmium0300.png}\\
        \legende{Corrected velocity, cost function: 0.23, number of iterations: 30000}
      \end{figure}
    \end{column}
  \end{columns}
\end{frame}
%%%%%%%%%%%%%%%%%%%%%%%%%%%%%%%%%%%%%%%%%%%%%%%%%%%%%%%%%%%%%%%%%%%%%%%%

	\section[]{Conclusion}
%14%%%%%%%%%%%%%%%%%%%%%%%%%%%%%%%%%%%%%%%%%%%%%%%%%%%%%%%%%%%%%%%%%%%%%
\begin{frame}
  \frametitle{Conclusion}

  \begin{block}{Sub-mesoscale tracers invertible to larger scales}
  \begin{itemize}  
    \item Sub-mesoscale tracers inversion to mesoscale velocity is feasible using Lyapunov exponents as a proxy. 
    \item High resolution Sea Surface Temperature or Ocean Color data are usefull to control ocean physics. 
  \end{itemize}
  \end{block}
  \begin{block}{Next}
  \begin{itemize}
    \item Quantify the error made on the estimated velocity
    \item Avoid the degradation of the corrected velocity with the number of iterations
  \end{itemize}
  \end{block}

  \begin{block}{Prospects}
  \begin{itemize}
    \item Assimilation of image in a coupled physico-biogeochemical model
    \item 
  \end{itemize}
  \end{block}
\end{frame}
%%%%%%%%%%%%%%%%%%%%%%%%%%%%%%%%%%%%%%%%%%%%%%%%%%%%%%%%%%%%%%%%%%%%%%%%

%15%%%%%%%%%%%%%%%%%%%%%%%%%%%%%%%%%%%%%%%%%%%%%%%%%%%%%%%%%%%%%%%%%%%%%
\begin{frame}
\begin{center}
Thank you for your attention
\end{center}
\end{frame}
%%%%%%%%%%%%%%%%%%%%%%%%%%%%%%%%%%%%%%%%%%%%%%%%%%%%%%%%%%%%%%%%%%%%%%%%

%%3%%%%%%%%%%%%%%%%%%%%%%%%%%%%%%%%%%%%%%%%%%%%%%%%%%%%%%%%%%%%%%%%%%%%%
%\begin{frame}
%  \frametitle{}
%  \begin{block}{ The objective is to explore the feasability of using submesoscale tracer information to controle ocean dynamic fields}
%   \begin{itemize}
%%    \item Use of Lyapunov exponents as a proxy to compare velocity fields and Chlorophyll or Sea Surface Temperature images
%    \item Comparison of FSLE and Chlorophyll or SST patterns (d'Ovidio et al, 2004)
%    \item Inversion of submesoscale FSLE (Finite-size Lyapunov Exponents) images to mesoscale velocity
%    \item Inversion of submesoscale SST images to mesoscale velocity
%%     \item Inversion of submesoscale FSLE images to mesoscale velocity
%
%  \end{itemize}
%  \end{block}
%\  
%
%%Velocity fields: Aviso altimeter Data\\
%%Tracer images: SST and Chlorphyll from MODIS captor\\
%%Region: Mediterranean Sea : from $4.8^oE$ to $8^oE$, from $38.2^oN$ to $40.^oN$ \\ 
%
%\end{frame}

%\begin{frame}
%  \frametitle{test Balthazar}
%  \begin{center}
%  \begin{tikzpicture}
%% les C et les H
%    \foreach \n/\a in {a/30,b/90,c/150,d/210,e/270,f/330}
%      {\node (\n) at (\a:1) {C};
%       \node (\n\n) at (\a:2) {H};}

%% les liaisons C - H
 %   \foreach \n in {a,b,c,d,e,f} \draw [thick] (\n)--(\n\n);

%%les liaisons simple entre C
%    \draw [thick](a)--(b);
%    \draw [thick] (c)--(d);
%    \draw [thick] (e)--(f);

%%les liaisons doubles entre C
%    \draw [double,thick] (b)--(c);
%    \draw [double,thick] (d)--(e);
%    \draw [double,thick] (f)--(a);
%  \end{tikzpicture}
%  \begin{block}{}
%    \centering
%    TADA...
%  \end{block}
%  \end{center}
%\end{frame}
%%assimilation of submesoscale obs into ocean models for the control of larger scales
%	%Is it feasible
%	%Can se use image proxies
%	%Are Lyapunov exponents a relaible proxy
%	%Can we make the link between altimetry and ocean color, physics and biogeochemistry
%%Two steps : 1 Are subemesoscale synthetic images invertilbes to larger scale ocean circulation
%%	     2 are submesoscale ocean color images invertibles to larger scale ocean circulation
%
%
%%	\section{Data set}
% %%4%%%%%%%%%%%%%%%%%%%%%%%%%%%%%%%%%%%%%%%%%%%%%%%%%%%%%%%%%%%%%%%%%%%%%
%\begin{frame}
%  \frametitle{Data set}
%  \begin{itemize}
%    \item \textbf{Region}: Mediterranean Sea, from $4.8^oE$ to $8^oE$, from $38.2^oN$ to $40.^oN$
%    \item \textbf{Time Range}: from 1998 to June 2009, 595 velocity maps
%    \item \textbf{Velocity fields}: AVISO altimeter data
%    \item \textbf{Resolution}: $1/8^o$, grid points: 26*17
%    \item \textbf{FSLE Resolution}: $1/48^o$, grid points: 119*86
%  \end{itemize}
%%  \vsapce{1cm}
%  \begin{itemize}
%    \item \textbf{SST field}: Data from MODIS captor, L2 product
%    \item \textbf{Resolution}: $1/100^o$
%  \end{itemize}
%\end{frame}
%
%	\section{Comparison between FSLE fields and SST}
%
%\begin{frame}
%  \frametitle{Method to detect filaments in SST image}
%  \begin{center}
%  \includegraphics[width=0.8\linewidth]{./scheme.jpg}
%  \end{center}
%\end{frame}
%
%%7%%%%%%%%%%%%%%%%%%%%%%%%%%%%%%%%%%%%%%%%%%%%%%%%%%%%%%%%%%%%%%%%%%%%%
%\begin{frame}[plain]
%  \frametitle{Method to detect filaments in SST image}
%%  MODIS SST $\rightarrow$ Lanczos Filter $\rightarrow$ Gradiant $\rightarrow$ Decrease resolution $\rightarrow$ Binarization $\rightarrow$ Filaments
%  \begin{columns}
%    \begin{column}{0.5\textwidth}
%      \begin{figure}%[!htbp]
%        \includegraphics[width=4.50cm]{../Lyap/Plot/SST/A2004184123500_L2_LAC_SST_record_bin1.jpg}\\
%        \legende{SST without filtering}
%      \end{figure}
%    \end{column}
%    \begin{column}{0.5\textwidth}
%      \begin{figure}%[!htbp]
%        \includegraphics[width=4.50cm]{../Lyap/Plot/SST/A2004184123500_L2_LAC_SST_record_bin4.jpg}\\
%        \legende{SST with $\lambda$=10 Lanczos filter}
%      \end{figure}
%    \end{column}
%  \end{columns}
%  \begin{columns}
%    \begin{column}{0.5\textwidth}
%      \begin{figure}
%        \includegraphics[width=4.50cm]{../Lyap/Plot/SST/A2004184123500_L2_LAC_SST_record_bin6.jpg}\\
%        \legende{SST with $\lambda$=15 Lanczos filter}
%      \end{figure}
%    \end{column}
%    \begin{column}{0.5\textwidth}
%      \begin{figure}%[!htbp]
%        \includegraphics[width=4.50cm]{../Lyap/Plot/SST/A2004184123500_L2_LAC_SST_record_bin9.jpg}\\
%        \legende{SST $\lambda$=25 Lanczos filter}
%      \end{figure}
%    \end{column}
%  \end{columns}
%\end{frame}
%
%	\section{Inversion of submesoscale information}
%
%%7%%%%%%%%%%%%%%%%%%%%%%%%%%%%%%%%%%%%%%%%%%%%%%%%%%%%%%%%%%%%%%%%%%%%%
%\begin{frame}
%  \frametitle{Minimizing the cost function using a simulated annealing algorithm}
%  $\bullet$ Cost function: $J = \|\hat{\lambda}_{fsle}-\hat{\lambda}_{tracer}\| \times (1+log\left(\frac{\|u\|}{\|u_{aviso}\|} \right))$\\ 
%  \ 
%
%  $\bullet$ First estimate: Aviso velocity field \\ 
%  \ 
%
%  $\bullet$ Perturbation: background velocity error simulated by drawing from the Gaussian Probability distribution \\ 
%  \ 
%
%  $\bullet$ Amplitude of perturbation evolves with the cost function: $\gamma= \alpha \times(J-J_{0})$ \\ 
%  \ 
%
%  $\bullet$ Probability of accepting uphill move: $p=exp(-\delta J/T)$ with $T= \beta \times (J-J_{0})$ \\ 
%  	               
%\end{frame}
%
%%8%%%%%%%%%%%%%%%%%%%%%%%%%%%%%%%%%%%%%%%%%%%%%%%%%%%%%%%%%%%%%%%%%%%%%
%\begin{frame}
%  \frametitle{Comparison between Aviso velocity and the corrected one}
%  Correction of the mesoscale velocity minimizing the previous cost function: $\alpha=\frac{1}{300}$, $\beta=\frac{1}{10}$, $J_{0}=0$
%  \begin{columns}
%    \begin{column}{0.5\textwidth}
%      \begin{figure}%[!htbp]
%        \includegraphics[width=5.9cm]{../Lyap/Plot/aviso/aviso_velmap19904_med.jpg}\\
%        \legende{Aviso velocity, June 30, 2004, $J=0.32$}
%      \end{figure}
%    \end{column}
%    \begin{column}{0.5\textwidth}
%      \begin{figure} 
%        \includegraphics[width=5.9cm]{../Lyap/Plot/aviso/aviso_velmap19904_med37133_ch2.jpg}\\
%        \legende{Corrected velocity, $J=0.23$}
%      \end{figure}
%    \end{column}
%  \end{columns}
%\end{frame}
%
%%8%%%%%%%%%%%%%%%%%%%%%%%%%%%%%%%%%%%%%%%%%%%%%%%%%%%%%%%%%%%%%%%%%%%%%
%\begin{frame}[plain]
%  \frametitle{Comparison between Aviso velocity and the corrected one}
%%  MODIS SST $\rightarrow$ Lanczos Filter $\rightarrow$ Gradiant $\rightarrow$ Decrease resolution $\rightarrow$ Binarization $\rightarrow$ Filaments
%  \begin{columns}
%    \begin{column}{0.5\textwidth}
%      \begin{figure}%[!htbp]
%        \includegraphics[width=4.5cm]{../Lyap/Plot/19904_med/pfsle_48_stat_reg_19904_med.jpg}\\
%        \legende{FSLE from Aviso velocity, June 30, 2004}
%      \end{figure}
%    \end{column}
%    \begin{column}{0.5\textwidth}
%      \begin{figure}%[!htbp]
%        \includegraphics[width=4.5cm]{../Lyap/Plot/19904_med/pfsle_est37133_19904_med.jpg}\\
%        \legende{FSLE from Corrected velocity}
%      \end{figure}
%    \end{column}
%  \end{columns}
%  \begin{columns}
%    \begin{column}{0.5\textwidth}
%      \begin{figure}
%        \includegraphics[width=4.5cm]{../Lyap/Plot/SST/A2004184123500_L2_LAC_SST_record_bin6.jpg}\\
%        \legende{Filaments from tracer}
%      \end{figure}
%    \end{column}
%  \end{columns}
%\end{frame}
%
%
%
%%\begin{frame}[plain]
%%  \frametitle{Method to detect filaments in SST image}
%%%  MODIS SST $\rightarrow$ Lanczos Filter $\rightarrow$ Gradiant $\rightarrow$ Decrease resolution $\rightarrow$ Binarization $\rightarrow$ Filaments
%%  \begin{columns}
%%    \begin{column}{0.5\textwidth}
%%      \begin{figure}%[!htbp]
%%        \includegraphics[width=4.50cm]{../Lyap/Plot/SST/A2004184123500_L2_LAC_SST_record_bin1.jpg}\\
%%        \legende{SST not filtered}
%%      \end{figure}
%%    \end{column}
%%    \begin{column}{0.5\textwidth}
%%      \begin{figure}%[!htbp]
%%        \includegraphics[width=4.50cm]{../Lyap/Plot/SST/A2004184123500_L2_LAC_SST_record_bin4.jpg}\\
%%        \legende{SST filtered with $\lambda$=10}
%%      \end{figure}
%%    \end{column}
%%  \end{columns}
%%  \begin{columns}
%%    \begin{column}{0.5\textwidth}
%%      \begin{figure}
%%        \includegraphics[width=4.50cm]{../Lyap/Plot/SST/A2004184123500_L2_LAC_SST_record_bin6.jpg}\\
%%        \legende{SST filtered with $\lambda$=15}
%%      \end{figure}
%%    \end{column}
%%    \begin{column}{0.5\textwidth}
%%      \begin{figure}%[!htbp]
%%        \includegraphics[width=4.50cm]{../Lyap/Plot/SST/A2004184123500_L2_LAC_SST_record_bin9.jpg}\\
%%        \legende{SST filtered with $\lambda$=25}
%%      \end{figure}
%%    \end{column}
%%  \end{columns}
%%\end{frame}
%
%
%
%
%
%
%%  \begin{columns}
%%    \begin{column}{0.5\textwidth}
%%      \begin{figure}%[!htbp]
%%        \includegraphics[width=5cm]{../Lyap/Plot/19001/fsle_24_stat_reg_0019001.jpg}\\
%%        \legende{FSLE, Pomme area, January 09, 2002}
%%      \end{figure}
%%    \end{column}
%%    \begin{column}{0.5\textwidth}
%%      \begin{figure} 
%%        \includegraphics[width=6cm]{../Lyap/Plot/Jiter/Jiter_log_19001_pomme.png}\\
%%        \legende{Cost function as a function of Iteration, Pomme area, January 09, 2002}
%%      \end{figure}
%%    \end{column}
%%  \end{columns}
%%\end{frame}
%%
%%%5%%%%%%%%%%%%%%%%%%%%%%%%%%%%%%%%%%%%%%%%%%%%%%%%%%%%%%%%%%%%%%%%%%%%%
%%\begin{frame}[plain]{Previous results: Pomme area}
%%  \frametitle{Previous results: Pomme area}
%%  \begin{columns}
%%    \centering
%%    \begin{column}{0.5\textwidth}
%%      \centering
%%      \begin{figure}[!htbp]
%%        \includegraphics[scale=0.12]{../Lyap/Plot/Chloro/A2007154_L3m_DAY_CHL_chlor_a_4km_pomme.jpg}
%%%        \caption{Chlorophyll from Modis captor, 03 June 2007}    
%%      \end{figure}
%%    \end{column}
%%    \begin{column}{0.5\textwidth}  
%%      \begin{figure}[!htbp]
%%        \includegraphics[scale=0.12]{../Lyap/Plot/SST/A2007154_L3m_DAY_SST_4_pomme.jpg}
%%%        \caption{Sea Surface Temperature from Modis captor, June 03, 2007}
%%      \end{figure}
%%    \end{column}
%%  \end{columns}
%%  \begin{columns}
%%    \centering
%%    \begin{column}{0.5\textwidth}
%%      \begin{figure}[!htbp]    
%%        \includegraphics[scale=0.12]{../Lyap/Plot/Chloro/pA2007154_L3m_Day_CHL_chlor_a_4km_pomme.jpg}\\
%%        \legende{\hspace{1.cm}Binarized chlorophyll gradient from Modis captor, June 03, 2007\hspace{1.cm}} 
%%      \end{figure}
%%    \end{column}
%%    \begin{column}{0.5\textwidth} 
%%      \begin{figure}[!htbp]
%%        \includegraphics[scale=0.12]{../Lyap/Plot/SST/pA2007154_L3m_DAY_SST_4_pomme.jpg}\\
%%        \legende{\hspace{1.cm}Binarized SST gradient from Modis captor, June 03, 2007\hspace{1.cm}}
%%      \end{figure}
%%    \end{column}     
%%  \end{columns} 
%%\end{frame}
%%
%%%%6%%%%%%%%%%%%%%%%%%%%%%%%%%%%%%%%%%%%%%%%%%%%%%%%%%%%%%%%%%%%%%%%%%%%%%
%%\begin{frame}
%%  \frametitle{Previous results: Mediterranean area}
%%  \begin{columns}
%%    \centering
%%    \begin{column}{0.49\textwidth}
%%      \begin{figure}[!htbp]
%%%        \includegraphics[width=4cm]{../Lyap/Plot/19904_med/fsle_24_stat_reg_0019904meds.jpg}\\
%%        \includegraphics[width=5cm]{../Lyap/Plot/fsle_24_stat_reg_19218_meds.png}\\
%%        \legende{FSLE, Mediterranean Sea, small area, August 14, 2002}    
%%      \end{figure}
%%    \end{column}
%%    \begin{column}{0.49\textwidth}
%%      \begin{figure}[!htbp]
%%        \includegraphics[width=6cm]{../Lyap/Plot/Jiter/Jiter_log_19904_meds.png}\\
%%        \legende{Cost function as a function of number of iteration, T=200, neof=075}
%%      \end{figure}
%%    \end{column}
%%  \end{columns}
%%  \begin{block}{}
%%    Problems when minimizing the cost function
%%  \end{block}    
%%\end{frame}
%%
%%%7%%%%%%%%%%%%%%%%%%%%%%%%%%%%%%%%%%%%%%%%%%%%%%%%%%%%%%%%%%%%%%%%%%%%%%
%%\begin{frame}
%%  \frametitle{Resolution of FSLE divided by two in Mediterranean area}
%%  \begin{columns}
%%    \centering
%%    \begin{column}{0.49\textwidth}
%%      \begin{figure}[!htbp]
%%%        \includegraphics[width=4cm]{../Lyap/Plot/19904_med/fsle_24_stat_reg_19904meds.jpg}\\
%%        \includegraphics[width=5cm]{../Lyap/Plot/fsle_48_stat_reg_19218_meds.png}\\
%%        \legende{FSLE, Mediterranean Sea, small area, August 14, 2002}
%%      \end{figure}
%%    \end{column}
%%    \begin{column}{0.49\textwidth}
%%      \begin{figure}[!htbp]
%%        \includegraphics[width=6cm]{../Lyap/Plot/Jiter/Jiter_log_19218_0075medsT200.png}\\
%%        \legende{Cost function as a function of number of iteration, T=200, neof=075}
%%      \end{figure}
%%    \end{column}
%%  \end{columns}
%%  \begin{block}{}
%%    Problems when minimizing the cost function
%%  \end{block}
%%\end{frame}
%%
%%
%%%8%%%%%%%%%%%%%%%%%%%%%%%%%%%%%%%%%%%%%%%%%%%%%%%%%%%%%%%%%%%%%%%%%%%%%%%
%%\begin{frame}[plain]{Previous results: SST in Mediterranean area from MODIS}
%%  \frametitle{Previous results: SST in Mediterranean area from MODIS}
%%  \begin{columns}
%%    \centering
%%    \begin{column}{0.5\textwidth}
%%      \begin{figure}[!htbp]
%%        \includegraphics[scale=0.12]{../Lyap/Plot/SST/A2004184_L3m_DAY_SST_4_med.jpg}\\
%%%        \legende{Sea Surface temperature in the Mediterranean Sea from MODIS captor}
%%      \end{figure}
%%    \end{column}
%%    \begin{column}{0.5\textwidth}
%%      \begin{figure}[!htbp]
%%        \includegraphics[scale=0.12]{../Lyap/Plot/SST/A2004184123500_L2_LAC_SST.jpg}
%%%        \legende{Binarized Sea Surface Temperature}
%%      \end{figure}
%%    \end{column}
%%  \end{columns}
%%  \begin{columns}
%%    \centering
%%    \begin{column}{0.5\textwidth}
%%      \begin{figure}[!htbp]
%%        \includegraphics[scale=0.12]{../Lyap/Plot/SST/pA2004184_L3m_DAY_SST_4_med.jpg}\\
%%        \legende{Low Resolution}
%%      \end{figure}
%%    \end{column}
%%    \begin{column}{0.5\textwidth}
%%      \begin{figure}[!htbp]
%%        \includegraphics[scale=0.12]{../Lyap/Plot/SST/pA2004184123500_L2_LAC_SST.jpg}\\
%%        \legende{High Resolution}
%%      \end{figure}
%%    \end{column}
%%  \end{columns}
%%\end{frame}
%%
%%%9%%%%%%%%%%%%%%%%%%%%%%%%%%%%%%%%%%%%%%%%%%%%%%%%%%%%%%%%%%%%%%%%%%%%%%
%%\begin{frame}
%%  \frametitle{Tests from Didier Auroux}
%%  \begin{columns}
%%    \centering
%%    \begin{column}{0.5\textwidth}
%%      \begin{figure}[!htbp]
%%        \includegraphics[scale=0.19]{./pict/test_HR.jpg}\\
%%        \legende{SST in Mediterranean area}
%%      \end{figure}
%%    \end{column}
%%    \begin{column}{0.5\textwidth}
%%      \begin{figure}[!htbp]
%%        \includegraphics[scale=0.33]{./pict/test_didier.jpg}\\
%%        \legende{Test from Didier Auroux}
%%      \end{figure}
%%    \end{column}
%%  \end{columns}
%%\end{frame}
%%
%%
%%%10%%%%%%%%%%%%%%%%%%%%%%%%%%%%%%%%%%%%%%%%%%%%%%%%%%%%%%%%%%%%%%%%%%%%%%
%%\begin{frame}
%%  \frametitle{Previous difficulties}
%%  \begin{block}{}
%%    \begin{itemize}
%%      \item FSLE are not accurate near the coast
%%      \item Find the simulating annealing factors that make the minimization of the cost function possible
%%      \item Find an area with few clouds, presence of passive tracers and mesosale stirring
%%      \item Structures highlights by binarization not continuous
%%    \end{itemize}
%%  \end{block}
%%\end{frame}
%%
%%	\section{Studies on several areas}
%%
%%%11%%%%%%%%%%%%%%%%%%%%%%%%%%%%%%%%%%%%%%%%%%%%%%%%%%%%%%%%%%%%%%%%%%%%%%
%%\begin{frame}
%%  \frametitle{Leeuwin Current}
%%  \begin{columns}
%%    \centering
%%    \begin{column}{0.5\textwidth}
%%      \begin{figure}[!htbp]
%%        \includegraphics[scale=0.32]{../Lyap/Plot/fsle_24_stat_reg_19253_leeuwin.png}\\
%%        \legende{FSLE, Leeuwin Current}
%%      \end{figure}
%%    \end{column}
%%    \begin{column}{0.5\textwidth}
%%      \begin{figure}[!htbp]
%%        \includegraphics[scale=0.32]{../Lyap/Plot/SST/SST_L3m_DAY_4_2002261_leeuwin.png}\\
%%        \legende{SST, Leeuwin Current}
%%      \end{figure}
%%    \end{column}
%%  \end{columns}
%%  \begin{block}{}
%%    Mesoscale structures visible, no submesoscale activity found in the litterature\\
%%    Unaccurate in the vicinity of the coast
%%  \end{block}
%%\end{frame}
%%
%%%12%%%%%%%%%%%%%%%%%%%%%%%%%%%%%%%%%%%%%%%%%%%%%%%%%%%%%%%%%%%%%%%%%%%%%%
%%\begin{frame}
%%  \frametitle{Leeuwin Current}
%%  \begin{columns}
%%    \centering
%%    \begin{column}{0.5\textwidth}
%%      \begin{figure}[!htbp]
%%        \includegraphics[scale=0.32]{../Lyap/Plot/pfsle_24_stat_reg_19253_leeuwin.png}\\
%%        \legende{Binarized FSLE, Leeuwin Current}
%%      \end{figure}
%%    \end{column}
%%    \begin{column}{0.5\textwidth}
%%      \begin{figure}[!htbp]
%%        \includegraphics[scale=0.32]{../Lyap/Plot/SST/pSST_L3m_DAY_4_2002261_leeuwin.png}\\
%%        \legende{Binarized SST, Leeuwin Current}
%%      \end{figure}
%%    \end{column}
%%  \end{columns}
%%  \begin{block}{}
%%    Mesoscale structures visible, no submesoscale activy found in the litterature\\
%%    Difficulty with the vicinity of the coast
%%  \end{block}
%%\end{frame}
%%
%%
%%%13%%%%%%%%%%%%%%%%%%%%%%%%%%%%%%%%%%%%%%%%%%%%%%%%%%%%%%%%%%%%%%%%%%%%%%
%%\begin{frame}
%%  \frametitle{Californian Current}
%%
%%  \begin{columns}
%%    \centering
%%    \begin{column}{0.5\textwidth}
%%      \begin{figure}[!htbp]
%%        \includegraphics[scale=0.38]{../Lyap/Plot/fsle_24_stat_reg_19988_california.png}\\
%%        \legende{FSLE, Californian Current}
%%      \end{figure}
%%    \end{column}
%%    \begin{column}{0.5\textwidth}
%%      \begin{figure}[!htbp]
%%        \includegraphics[scale=0.38]{../Lyap/Plot/SST/SST_L3m_DAY_4_2004266_california.png}\\
%%        \legende{SST, Californian Current}
%%      \end{figure}
%%    \end{column}
%%  \end{columns}
%%  \begin{block}{}
%%    Coastal current, the filaments are too close to the coast.
%%  \end{block}
%%\end{frame}
%%
%%%14%%%%%%%%%%%%%%%%%%%%%%%%%%%%%%%%%%%%%%%%%%%%%%%%%%%%%%%%%%%%%%%%%%%%%%
%%\begin{frame}
%%  \frametitle{Californian Current}
%%
%%  \begin{columns}
%%    \centering
%%    \begin{column}{0.5\textwidth}
%%      \begin{figure}[!htbp]
%%        \includegraphics[scale=0.38]{../Lyap/Plot/SST/SST_L3m_DAY_4_2004266_california.png}\\
%%        \legende{SST, Californian Current}
%%      \end{figure}
%%    \end{column}
%%    \begin{column}{0.5\textwidth}
%%      \begin{figure}[!htbp]
%%        \includegraphics[scale=0.38]{../Lyap/Plot/SST/pSST_L3m_DAY_4_2004266_california.png}\\
%%        \legende{SST, Californian Current}
%%      \end{figure}
%%    \end{column}
%%  \end{columns}
%%\end{frame}
%%
%%%15%%%%%%%%%%%%%%%%%%%%%%%%%%%%%%%%%%%%%%%%%%%%%%%%%%%%%%%%%%%%%%%%%%%%%%
%%\begin{frame}
%%  \frametitle{South Atlantic, East of Tasmania}
%%
%%  \begin{columns}
%%    \centering
%%    \begin{column}{0.5\textwidth}
%%      \begin{figure}[!htbp]
%%        \includegraphics[scale=0.42]{../Lyap/Plot/fsle_24_stat_reg_20079_tasmania.png}\\
%%        \legende{FSLE, South Pacific }
%%      \end{figure}
%%    \end{column}
%%    \begin{column}{0.5\textwidth}
%%      \begin{figure}[!htbp]
%%        \includegraphics[scale=0.42]{../Lyap/Plot/SST/SST_L3m_DAY_4_2004358_tasmania.png}\\
%%        \legende{SST, South Pacific}
%%      \end{figure}
%%    \end{column}
%%  \end{columns}
%%
%%\end{frame}
%%
%%%16%%%%%%%%%%%%%%%%%%%%%%%%%%%%%%%%%%%%%%%%%%%%%%%%%%%%%%%%%%%%%%%%%%%%%%
%%\begin{frame}
%%  \frametitle{South Pacific, East of Tasmania}
%%
%%  \begin{columns}
%%    \centering
%%    \begin{column}{0.5\textwidth}
%%      \begin{figure}[!htbp]
%%        \includegraphics[scale=0.42]{../Lyap/Plot/pfsle_24_stat_reg_20079_tasmania.png}\\
%%        \legende{FSLE, South Pacific }
%%      \end{figure}
%%    \end{column}
%%    \begin{column}{0.5\textwidth}
%%      \begin{figure}[!htbp]
%%        \includegraphics[scale=0.42]{../Lyap/Plot/SST/pSST_L3m_DAY_4_2004358_tasmania.png}\\
%%        \legende{SST, South Pacific }
%%      \end{figure}
%%    \end{column}
%%  \end{columns}
%%
%%\end{frame}
%%
%%%17%%%%%%%%%%%%%%%%%%%%%%%%%%%%%%%%%%%%%%%%%%%%%%%%%%%%%%%%%%%%%%%%%%%%%%
%%\begin{frame}
%%  \frametitle{South Atlantic, East of Tasmania}
%%  \begin{columns}
%%    \centering
%%    \begin{column}{0.5\textwidth}
%%      \begin{figure}[!htbp]
%%        \includegraphics[scale=0.42]{../Lyap/Plot/SST/SST_L3m_DAY_4_2004358_tasmania.png}\\
%%        \legende{SST, South Pacific}
%%      \end{figure}
%%    \end{column}
%%    \begin{column}{0.5\textwidth}
%%      \begin{figure}[!htbp]
%%        \includegraphics[scale=0.42]{../Lyap/Plot/SST/pSST_L3m_DAY_4_2004358_tasmania.png}\\
%%        \legende{SST, South Pacific }
%%      \end{figure}
%%    \end{column}
%%  \end{columns}
%%\end{frame}
%%
%%%18%%%%%%%%%%%%%%%%%%%%%%%%%%%%%%%%%%%%%%%%%%%%%%%%%%%%%%%%%%%%%%%%%%%%%%
%%\begin{frame}
%%  \frametitle{South Atlantic, East of Tasmania}
%%  \begin{columns}
%%    \centering
%%    \begin{column}{0.5\textwidth}
%%      \begin{figure}[!htbp]
%%        \includegraphics[scale=0.42]{../Lyap/Plot/Chloro/CHL_L3m_DAY_4_2004358_tasmania.png}\\
%%        \legende{Chlorophyll, South Pacific}
%%      \end{figure}
%%    \end{column}
%%    \begin{column}{0.5\textwidth}
%%      \begin{figure}[!htbp]
%%        \includegraphics[scale=0.42]{../Lyap/Plot/Chloro/pCHL_L3m_DAY_4_2004358_tasmania.png}\\
%%        \legende{Chlorophyll, South Pacific}
%%      \end{figure}
%%    \end{column}
%%  \end{columns}
%%\end{frame}
%%
%%%19%%%%%%%%%%%%%%%%%%%%%%%%%%%%%%%%%%%%%%%%%%%%%%%%%%%%%%%%%%%%%%%%%%%%%%
%%\begin{frame}
%%  \frametitle{South Pacific, East of Tasmania}
%%  \begin{columns}
%%    \centering
%%    \begin{column}{0.5\textwidth}
%%      \begin{figure}[!htbp]
%%        \includegraphics[scale=0.42]{../Lyap/Plot/pfsle_24_stat_reg_20079_tasmania.png}\\
%%        \legende{FSLE, South Pacific }
%%      \end{figure}
%%    \end{column}
%%    \begin{column}{0.5\textwidth}
%%      \begin{figure}[!htbp]
%%        \includegraphics[scale=0.42]{../Lyap/Plot/Chloro/pCHL_L3m_DAY_4_2004358_tasmania.png}\\
%%        \legende{Chlorophyll, South Pacific }
%%      \end{figure}
%%    \end{column}
%%  \end{columns}
%%\end{frame}
%%
%%%20%%%%%%%%%%%%%%%%%%%%%%%%%%%%%%%%%%%%%%%%%%%%%%%%%%%%%%%%%%%%%%%%%%%%%%
%%\begin{frame}
%%  \frametitle{South Pacific, East of Tasmania}
%%  \begin{columns}
%%    \centering
%%    \begin{column}{0.5\textwidth}
%%      \begin{figure}[!htbp]
%%        \includegraphics[scale=0.42]{../Lyap/Plot/SST/pSST_L3m_DAY_4_2004358_tasmania.png}\\
%%        \legende{SST, South Pacific }
%%      \end{figure}
%%    \end{column}
%%    \begin{column}{0.5\textwidth}
%%      \begin{figure}[!htbp]
%%        \includegraphics[scale=0.42]{../Lyap/Plot/Chloro/pCHL_L3m_DAY_4_2004358_tasmania.png}\\
%%        \legende{Chlorophyll, South Pacific }
%%      \end{figure}
%%    \end{column}
%%  \end{columns}
%%\end{frame}
%%
%%%21%%%%%%%%%%%%%%%%%%%%%%%%%%%%%%%%%%%%%%%%%%%%%%%%%%%%%%%%%%%%%%%%%%%%%%
%%\begin{frame}
%%  \frametitle{TO DO LIST}
%%  \begin{itemize}
%%    \item Inversion of FSLE images on Tasmania area
%%    \item Meeting with Francesco d'Ovidio
%%    \item Binarization method need to be improved
%%    \item Singularity Exponents?
%%    \item Choice of a model to study
%%    \item Meeting with Marina L�vy
%%  \end{itemize}
%%\end{frame}
%%
%%%%
%%%%%%%%%%%CHOICE OF AN AREA
%%%%%%%%%%%BINARIZATION IS POSSIBLE
%%%
%%%





\end{document}
